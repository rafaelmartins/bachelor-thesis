\chapter{Apresentação dos processos referenciais industriais
    \label{cap:processos-referenciais}}

\section{Introdução}

Neste capítulo são apresentados os 14 processos referenciais utilizados
como base para este trabalho. Estes processos referenciais são capazes de se
aproximar de uma grande quantidade dos processos reais encontrados na indústria,
sendo interessantes para o estudo e até mesmo para a sintonia dos controladores
encontrados nas plantas industriais. Estes processos já foram amplamente destacados
na literatura \cite{Isermann} \cite{AstromHagglund}.

Vale ressaltar que, como se trata de um projeto didático, nem todos os processos
conseguem ser identificados e controlados utilizando-se os métodos propostos
neste trabalho, e que a análise da eficiência dos métodos faz parte do aprendizado
do aluno que utiliza-se deste software como ferramenta de estudo dos controladores
\acs{PID}.

\section{Processos referenciais industriais}

\subsection{Processo de primeira ordem}
    
    O processo de primeira ordem é o mais simples de todo, sendo bastante encontrado
    nas plantas industriais. É capaz de representar uma boa quantidade de motores,
    entre outros processos.
    
    Devido às suas caracteristicas dinâmicas (definidas pelo ganho estático $k$ e
    pela constante de tempo $\tau$), pelo seu curto tempo de subida, este processo
    é dificil de se identificar através de métodos que utilizem a curva de reação.
    
    \begin{equation}
        G_p(s) = \frac{k}{(1+\tau s)}
    \end{equation}
    
    \begin{center}
        \includegraphics[width=\textwidth]{imagens/cap2_model1_1.eps}
        \captionof{figure}{Processo de primeira ordem}
    \end{center}

\subsection{Processo de segunda ordem}
    
    O processo de segunda ordem é também bastante difundido na industria. Possui
    características dinâmicas definidas pelo ganho estático $k$ e os tempos $T_1$
    e $T_2$.
    
    Por não apresentar um crescimento monotônico, este processo é mais facil de ser
    identificado a partir dos métodos de identificação utilizando a curva de reação
    do que o processo referencial de primeira ordem.
    
    \begin{equation}
        G_p(s) = \frac{k}{(1+T_1 s)(1+T_2 s)}
    \end{equation}
    
    \begin{center}
        \includegraphics[width=\textwidth]{imagens/cap2_model2_1.eps}
        \captionof{figure}{Processo de segunda ordem}
    \end{center}

\subsection{Processo de segunda ordem de fase não-mínima}

    O processo de segunda ordem de fase não-minima tem uma dinâmica diferente dos
    outros processos já apresentados. Este processo apresenta resposta inversa à
    entrada, quando recebendo um sinal em forma de degrau, até um certo limite,
    passando a rastrear o sinal de entrada (referência).
    
    Não apresenta grandes dificuldades para a identificação utilizando os métodos
    de identificação por curva de reação, dependendo dos parâmetros do processo,
    que são o ganho estático $k$ e os tempos $T_1$ e $T_2$.

    \begin{equation}
        G_p(s) = \frac{k(1-T_1 s)}{(1+T_1 s)(1+T_2 s)}
    \end{equation}

    \begin{center}
        \includegraphics[width=\textwidth]{imagens/cap2_model3_1.eps}
        \captionof{figure}{Processo de segunda ordem de fase não-mínima}
    \end{center}

\subsection{Processo de terceira ordem com tempo morto ajustável}
    
    O processo de terceira ordem pode assumir várias dinâmicas diferentes, dependendo
    dos parâmetros do sistema, que são o ganho estático $k$, os tempos $T_1$, $T_2$,
    $T_3$ e $T_4$ e o tempo morto $T_t$.
    
    Processos com tempo morto tendem a ser mais faceis de identificar utilizando-se
    os metodos de identificação através da curva de reação, porém são mais dificeis
    de se controlar com controladores \acs{PID} convencionais.
    
    \begin{equation}
        G_p(s) = \frac{k(1+T_4 s)}{(1+T_1 s)(1+T_2 s)(1+T_3 s)} e^{-T_t s}
    \end{equation}
    
    \begin{center}
        \includegraphics[width=\textwidth]{imagens/cap2_model4_1.eps}
        \captionof{figure}{Processo de terceira ordem com tempo morto ajustável}
    \end{center}

\subsection{Processo de pólos múltiplos e iguais}

    O processo de pólos múltiplos e iguais possui dinâmica variável, dependendo
    do parâmetro $n$, que representa a ordem do sistema.
    
    O aumento da ordem do sistema tende a facilitar a identificação através dos
    métodos de identificação por curva de reação.
    
    Os valores de $n$ variam de 1 a 4, tipicamente.
    
    \begin{equation}
        G_p(s) = \frac{1}{(s+1)^n}
    \end{equation}

    \begin{center}
        \includegraphics[width=0.75\textwidth]{imagens/cap2_model5_1.eps}
        \captionof{figure}{Processo de pólos múltiplos e iguais - $n = 1$}
    \end{center}
    
    \begin{center}
        \includegraphics[width=0.75\textwidth]{imagens/cap2_model5_2.eps}
        \captionof{figure}{Processo de pólos múltiplos e iguais - $n = 2$}
    \end{center}
    
    \begin{center}
        \includegraphics[width=0.75\textwidth]{imagens/cap2_model5_3.eps}
        \captionof{figure}{Processo de pólos múltiplos e iguais - $n = 3$}
    \end{center}
    
    \begin{center}
        \includegraphics[width=0.75\textwidth]{imagens/cap2_model5_4.eps}
        \captionof{figure}{Processo de pólos múltiplos e iguais - $n = 4$}
    \end{center}

\subsection{Processo de quarta ordem}
    
    O processo de quarta ordem tem dinâmica definida pelo parametro $\alpha$,
    sendo facilmente controlavel para valores pequenos de $\alpha$.
    
    Para $\alpha = 1$, este processo se comporta como o anterior, para $n = 4$.
    
    \begin{equation}
        G_p(s) = \frac{1}{(s+1)(\alpha s+1)(\alpha ^2 s+1)(\alpha ^3 s+1)}
    \end{equation}

    \begin{center}
        \includegraphics[width=0.75\textwidth]{imagens/cap2_model6_1.eps}
        \captionof{figure}{Processo de quarta ordem - $\alpha = 0.2$}
    \end{center}
    
    \begin{center}
        \includegraphics[width=0.75\textwidth]{imagens/cap2_model6_2.eps}
        \captionof{figure}{Processo de quarta ordem - $\alpha = 0.5$}
    \end{center}

\subsection{Processo com três pólos iguais e zero no semi-plano direito}

    \begin{center}
        \includegraphics[width=\textwidth]{imagens/cap2_model7_1.eps}
        \captionof{figure}{Processo com três pólos iguais e zero no semi-plano direito}
    \end{center}

\subsection{Processo de primeira ordem com tempo morto}

    \begin{center}
        \includegraphics[width=\textwidth]{imagens/cap2_model8_1.eps}
        \captionof{figure}{Processo de primeira ordem com tempo morto}
    \end{center}

\subsection{Processo de segunda ordem com tempo morto}

    \begin{center}
        \includegraphics[width=\textwidth]{imagens/cap2_model9_1.eps}
        \captionof{figure}{Processo de segunda ordem com tempo morto}
    \end{center}

\subsection{Processo com características dinâmicas assimétricas}

    \begin{center}
        \includegraphics[width=\textwidth]{imagens/cap2_model10_1.eps}
        \captionof{figure}{Processo com características dinâmicas assimétricas}
    \end{center}

\subsection{Processo condicionalmente estável}

    \begin{center}
        \includegraphics[width=\textwidth]{imagens/cap2_model11_1.eps}
        \captionof{figure}{Processo condicionalmente estável}
    \end{center}

\subsection{Processo oscilatório}

    \begin{center}
        \includegraphics[width=\textwidth]{imagens/cap2_model12_1.eps}
        \captionof{figure}{Processo oscilatório}
    \end{center}

\subsection{Processo instável}

    \begin{center}
        \includegraphics[width=\textwidth]{imagens/cap2_model13_1.eps}
        \captionof{figure}{Processo instável}
    \end{center}

\subsection{Processo de primeira ordem mais tempo morto com a presença de integrador}

    \begin{center}
        \includegraphics[width=\textwidth]{imagens/cap2_model14_1.eps}
        \captionof{figure}{Processo de primeira ordem mais tempo morto com a presença de integrador}
    \end{center}

\section{Conclusões parciais}

    TODO.
