\chapter{Apresentação dos processos referenciais industriais
    \label{cap:processos-referenciais}}

Neste capítulo são apresentados os 14 processos referenciais utilizados
como base para o trabalho.

\section{Processos referenciais industriais}

\subsection{Processo de primeira ordem}

    \begin{center}
        \includegraphics[width=\textwidth]{imagens/cap2_model1_1.eps}
        \captionof{figure}{Processo de primeira ordem}
    \end{center}

\subsection{Processo de segunda ordem}

    \begin{center}
        \includegraphics[width=\textwidth]{imagens/cap2_model2_1.eps}
        \captionof{figure}{Processo de segunda ordem}
    \end{center}

\subsection{Processo de segunda ordem de fase não-mínima}

    \begin{center}
        \includegraphics[width=\textwidth]{imagens/cap2_model3_1.eps}
        \captionof{figure}{Processo de segunda ordem de fase não-mínima}
    \end{center}

\subsection{Processo de terceira ordem com tempo morto ajustável}

    \begin{center}
        \includegraphics[width=\textwidth]{imagens/cap2_model4_1.eps}
        \captionof{figure}{Processo de terceira ordem com tempo morto ajustável}
    \end{center}

\subsection{Processo de pólos múltiplos e iguais}

    \begin{center}
        \includegraphics[width=\textwidth]{imagens/cap2_model5_1.eps}
        \captionof{figure}{Processo de pólos múltiplos e iguais}
    \end{center}

\subsection{Processo de quarta ordem}

    \begin{center}
        \includegraphics[width=\textwidth]{imagens/cap2_model6_1.eps}
        \captionof{figure}{Processo de quarta ordem}
    \end{center}

\subsection{Processo com três pólos iguais e zero no semi-plano direito}

    \begin{center}
        \includegraphics[width=\textwidth]{imagens/cap2_model7_1.eps}
        \captionof{figure}{Processo com três pólos iguais e zero no semi-plano direito}
    \end{center}

\subsection{Processo de primeira ordem com tempo morto}

    \begin{center}
        \includegraphics[width=\textwidth]{imagens/cap2_model8_1.eps}
        \captionof{figure}{Processo de primeira ordem com tempo morto}
    \end{center}

\subsection{Processo de segunda ordem com tempo morto}

    \begin{center}
        \includegraphics[width=\textwidth]{imagens/cap2_model9_1.eps}
        \captionof{figure}{Processo de segunda ordem com tempo morto}
    \end{center}

\subsection{Processo com características dinâmicas assimétricas}

    \begin{center}
        \includegraphics[width=\textwidth]{imagens/cap2_model10_1.eps}
        \captionof{figure}{Processo com características dinâmicas assimétricas}
    \end{center}

\subsection{Processo condicionalmente estável}

    \begin{center}
        \includegraphics[width=\textwidth]{imagens/cap2_model11_1.eps}
        \captionof{figure}{Processo condicionalmente estável}
    \end{center}

\subsection{Processo oscilatório}

    \begin{center}
        \includegraphics[width=\textwidth]{imagens/cap2_model12_1.eps}
        \captionof{figure}{Processo oscilatório}
    \end{center}

\subsection{Processo instável}

    \begin{center}
        \includegraphics[width=\textwidth]{imagens/cap2_model13_1.eps}
        \captionof{figure}{Processo instável}
    \end{center}

\subsection{Processo de primeira ordem mais tempo morto com a presença de integrador}

    \begin{center}
        \includegraphics[width=\textwidth]{imagens/cap2_model14_1.eps}
        \captionof{figure}{Processo de primeira ordem mais tempo morto com a presença de integrador}
    \end{center}

\section{Conclusões parciais}

    TODO.
