\chapter{Os processos referenciais industriais
    \label{cap:processos-referenciais}}

\section{Introdução}

Neste capítulo são apresentados os 14 processos referenciais utilizados
como base para este trabalho. Estes processos referenciais são capazes de se
aproximar de uma grande quantidade dos processos reais encontrados na indústria,
sendo interessantes para o estudo e até mesmo para a sintonia dos controladores
encontrados nas plantas industriais. Estes processos já foram amplamente destacados
na literatura \cite{Isermann:1981:DCS:539455} \cite{ast+hagg100}.

Vale ressaltar que, como se trata de um projeto didático, nem todos os processos
conseguem ser identificados e controlados utilizando-se os métodos propostos
neste trabalho, e que a análise da eficiência dos métodos faz parte do aprendizado
do aluno que utiliza-se deste software como ferramenta de estudo dos controladores
\acs{PID}.

\section{Processos referenciais industriais}

\subsection{Processo de primeira ordem}
    
    O processo de primeira ordem é o mais simples de todo, sendo bastante encontrado
    nas plantas industriais. É capaz de representar uma boa quantidade de motores,
    pequenos entre outros processos.
    
    Devido às suas caracteristicas dinâmicas (definidas pelo ganho estático $k$ e
    pela constante de tempo $\tau$), pelo seu curto tempo de subida, este processo
    é dificil de se identificar através de métodos que utilizem a curva de reação.
    A resposta ao degrau para o processo de primeira ordem encontra-se na figura
    \ref{cap2_1}
    
    \begin{equation}
        G_p(s) = \frac{k}{(1+\tau s)}
    \end{equation}
    
    \begin{center}
        \includegraphics[width=\textwidth]{imagens/cap2_model1_1.eps}
        \captionof{figure}{Processo de primeira ordem}
    \label{cap2_1}
    \end{center}

\subsection{Processo de segunda ordem}
    
    O processo de segunda ordem é também bastante difundido na industria. Possui
    características dinâmicas definidas pelo ganho estático $k$ e as constantes de
    tempo $T_1$ e $T_2$.
    
    Este processo é mais facil de servidentificado a partir dos métodos de
    identificação utilizando a curva de reação do que o processo referencial de
    primeira ordem. A resposta ao degrau para o processo de segunda ordem pode
    ser vista na figura \ref{cap2_2}
    
    \begin{equation}
        G_p(s) = \frac{k}{(1+T_1 s)(1+T_2 s)}
    \end{equation}
    
    \begin{center}
        \includegraphics[width=\textwidth]{imagens/cap2_model2_1.eps}
        \captionof{figure}{Processo de segunda ordem}
    \label{cap2_2}
    \end{center}

\subsection{Processo de segunda ordem de fase não-mínima}

    O processo de segunda ordem de fase não-minima tem uma dinâmica diferente dos
    outros processos já apresentados. Este processo apresenta resposta inversa à
    entrada, quando recebendo um sinal em forma de degrau, até um certo limite,
    passando a rastrear o sinal de entrada (referência). A fase não-mínima é
    determinada pelo zero no semiplano da direita.
    
    Não apresenta grandes dificuldades para a identificação utilizando os métodos
    de identificação por curva de reação, dependendo dos parâmetros do processo,
    que são o ganho estático $k$ e as constantes de tempo $T_1$ e $T_2$. A resposta
    ao degrau para o processo de segunda ordem de fase não-mínima está na figura
    \ref{cap2_3}.

    \begin{equation}
        G_p(s) = \frac{k(1-T_1 s)}{(1+T_1 s)(1+T_2 s)}
    \end{equation}

    \begin{center}
        \includegraphics[width=\textwidth]{imagens/cap2_model3_1.eps}
        \captionof{figure}{Processo de segunda ordem de fase não-mínima}
        \label{cap2_3}
    \end{center}

\subsection{Processo de terceira ordem com tempo morto ajustável}
    
    O processo de terceira ordem pode assumir várias dinâmicas diferentes, dependendo
    dos parâmetros do sistema, que são o ganho estático $k$, os tempos $T_1$, $T_2$,
    $T_3$ e $T_4$ e o tempo morto $T_t$.
    
    Processos com tempo morto tendem a ser mais faceis de identificar utilizando-se
    os metodos de identificação através da curva de reação, porém são mais dificeis
    de se controlar com controladores \acs{PID} convencionais, pois quanto maior
    o tempo morto, maior a razão de controlabilidade ($\alpha = \frac{L}{\tau}$).
    A resposta ao degrau para o processo de terceira ordem com tempo morto ajustável
    está na figura \ref{cap2_4}.
    
    \begin{equation}
        G_p(s) = \frac{k(1+T_4 s)}{(1+T_1 s)(1+T_2 s)(1+T_3 s)} e^{-T_t s}
    \end{equation}
    
    \begin{center}
        \includegraphics[width=\textwidth]{imagens/cap2_model4_1.eps}
        \captionof{figure}{Processo de terceira ordem com tempo morto ajustável}
        \label{cap2_4}
    \end{center}

\subsection{Processo de pólos múltiplos e iguais}

    O processo de pólos múltiplos e iguais possui dinâmica variável, dependendo
    do parâmetro $n$, que representa a ordem do sistema.
    
    O aumento da ordem do sistema tende a facilitar a identificação através dos
    métodos de identificação por curva de reação.
    
    Os valores de $n$ variam de 1 a 4, tipicamente. As figuras \ref{cap2_5},
    \ref{cap2_6}, \ref{cap2_7} e \ref{cap2_8} apresentam os gráficos para
    $n$ igual a 1, 2, 3 e 4.
    
    \begin{equation}
        G_p(s) = \frac{1}{(s+1)^n}
    \end{equation}

    \begin{center}
        \includegraphics[width=0.75\textwidth]{imagens/cap2_model5_1.eps}
        \captionof{figure}{Processo de pólos múltiplos e iguais - $n = 1$}
        \label{cap2_5}
    \end{center}
    
    \begin{center}
        \includegraphics[width=0.75\textwidth]{imagens/cap2_model5_2.eps}
        \captionof{figure}{Processo de pólos múltiplos e iguais - $n = 2$}
        \label{cap2_6}
    \end{center}
    
    \begin{center}
        \includegraphics[width=0.75\textwidth]{imagens/cap2_model5_3.eps}
        \captionof{figure}{Processo de pólos múltiplos e iguais - $n = 3$}
        \label{cap2_7}
    \end{center}
    
    \begin{center}
        \includegraphics[width=0.75\textwidth]{imagens/cap2_model5_4.eps}
        \captionof{figure}{Processo de pólos múltiplos e iguais - $n = 4$}
        \label{cap2_8}
    \end{center}

\subsection{Processo de quarta ordem}
    
    O processo de quarta ordem tem dinâmica definida pelo parametro $\alpha$,
    sendo facilmente controlavel para valores pequenos de $\alpha$, entre 0 e 1.
    
    Para $\alpha = 1$, este processo se comporta como o anterior, para $n = 4$.
    
    As figuras \ref{cap2_9} e \ref{cap2_10} mostram o comportamento deste
    processo para $\alpha$ igual a 0.2 e 0.5, respectivamente
    
    \begin{equation}
        G_p(s) = \frac{1}{(s+1)(\alpha s+1)(\alpha ^2 s+1)(\alpha ^3 s+1)}
    \end{equation}

    \begin{center}
        \includegraphics[width=0.75\textwidth]{imagens/cap2_model6_1.eps}
        \captionof{figure}{Processo de quarta ordem - $\alpha = 0.2$}
        \label{cap2_9}
    \end{center}
    
    \begin{center}
        \includegraphics[width=0.75\textwidth]{imagens/cap2_model6_2.eps}
        \captionof{figure}{Processo de quarta ordem - $\alpha = 0.5$}
        \label{cap2_10}
    \end{center}

\subsection{Processo com três pólos iguais e zero no semi-plano direito}

    O processo com três pólos iguais e zero no semi-plano direito do plano $s$
    possui dinâmica definida pelo parametro $\alpha$. A presença de um
    zero no semi-plano direito gera uma fase não-minima no sistema. Quanto maior
    o valor de $\alpha$, maior a dificuldade para se controlar o processo
    utilizando-se um controlador \acs{PID} convencional.
    
    Por possuir tempo morto, pode ser considerado um processo facil de identificar
    utilizando-se os métodos de identificação por curva de reação.
    
    As figuras \ref{cap2_11}, \ref{cap2_12}, \ref{cap2_13} e \ref{cap2_14}
    mostram o comportamento do processo para diferentes valores de $\alpha$.

    \begin{equation}
        G_p(s) = \frac{1-\alpha s}{(s+1)^3}
    \end{equation}

    \begin{center}
        \includegraphics[width=0.75\textwidth]{imagens/cap2_model7_1.eps}
        \captionof{figure}{Processo com três pólos iguais e zero no semi-plano direito - $\alpha = 0.2$}
        \label{cap2_11}
    \end{center}

    \begin{center}
        \includegraphics[width=0.75\textwidth]{imagens/cap2_model7_2.eps}
        \captionof{figure}{Processo com três pólos iguais e zero no semi-plano direito - $\alpha = 0.5$}
        \label{cap2_12}
    \end{center}
    
    \begin{center}
        \includegraphics[width=0.75\textwidth]{imagens/cap2_model7_3.eps}
        \captionof{figure}{Processo com três pólos iguais e zero no semi-plano direito - $\alpha = 1$}
        \label{cap2_13}
    \end{center}
    
    \begin{center}
        \includegraphics[width=0.75\textwidth]{imagens/cap2_model7_4.eps}
        \captionof{figure}{Processo com três pólos iguais e zero no semi-plano direito - $\alpha = 5$}
        \label{cap2_14}
    \end{center}

\subsection{Processo de primeira ordem com tempo morto}

    O processo de primeira ordem com tempo morto é considerado um processo
    clássico, e é bastante utilizado no estudo de controladores \acs{PID}. O
    processo possui dinâmica definida pela constante de tempo $\tau$ e pelo
    tempo morto $L$. Neste trabalho definimos o valor do tempo morto $L$ como 1.
    
    Assim como a maioria dos processos com tempo morto, pode ser considerado
    fácil de identificar utilizando métodos de curva de reação e complicado para
    controlar utilizando controladores \acs{PID} comuns.

    \begin{equation}
        G_p(s) = \frac{1}{(\tau s +1)}e^{-s}
    \end{equation}
    
    Nos gráficos abaixo existe oscilação no tempo morto, devido à aproximação de
    padè utilizada no software. Os graficos foram gerados para uma aproximação
    de quinta ordem.

    \begin{center}
        \includegraphics[width=0.75\textwidth]{imagens/cap2_model8_1.eps}
        \captionof{figure}{Processo de primeira ordem com tempo morto - $\tau = 0.5$}
    \end{center}

    \begin{center}
        \includegraphics[width=0.75\textwidth]{imagens/cap2_model8_2.eps}
        \captionof{figure}{Processo de primeira ordem com tempo morto - $\tau = 2$}
    \end{center}
    
    \begin{center}
        \includegraphics[width=0.75\textwidth]{imagens/cap2_model8_3.eps}
        \captionof{figure}{Processo de primeira ordem com tempo morto - $\tau = 10$}
    \end{center}

\subsection{Processo de segunda ordem com tempo morto}
    
    O Processo de segunda ordem com tempo morto é similar ao de primeira ordem
    com tempo morto, com diferenças de dinâmica dadas pela diferença de ordem
    dos processos e pela constante de tempo $\tau$, que possui os seguintes valores,
    típicamente: $0; 0,1; 0,2; 0,5; 2; 5; 10$. Para as figuras abaixo
    (\ref{cap2_18} e \ref{cap2_19})foram utilizados valores de $\tau$ iguais
    a 2 e 5.
    
    \begin{equation}
        G_p(s) = \frac{1}{(\tau s +1)^2}e^{-s}
    \end{equation}

    \begin{center}
        \includegraphics[width=0.75\textwidth]{imagens/cap2_model9_1.eps}
        \captionof{figure}{Processo de segunda ordem com tempo morto - $\tau = 2$}
        \label{cap2_18}
    \end{center}
    
    \begin{center}
        \includegraphics[width=0.75\textwidth]{imagens/cap2_model9_2.eps}
        \captionof{figure}{Processo de segunda ordem com tempo morto - $\tau = 5$}
        \label{cap2_19}
    \end{center}

\subsection{Processo com características dinâmicas assimétricas}
    
    O processo com características dinâmicas assimétricas possui duas características
    dinâmicas distintas, sendo uma rápida, com ganho estático igual a 1 e constante
    de tempo igual a 1, e outra lenta, com um ganho estático igual a 10.
    
    Este processo é dificil de controlar utilizando se controladores \acs{PID}
    simples e identificando o processo com metodos de curva de reação, por conta das
    características assimétricas do processo. Vide figura \ref{cap2_20}.
    
    \begin{equation}
        G_p(s) = \frac{100}{(s+10)^2}\left ( \frac{1}{s+1} + \frac{0,5}{s+0,05} \right )
    \end{equation}
    
    \begin{center}
        \includegraphics[width=\textwidth]{imagens/cap2_model10_1.eps}
        \captionof{figure}{Processo com características dinâmicas assimétricas}
        \label{cap2_20}
    \end{center}

\subsection{Processo condicionalmente estável}
    
    O processo condicionalmente estável pode se comportar como um processo estável
    ou instável, dependendo do ajuste do ponto de operação dos controladores.
    Vide figura \ref{cap2_21}.
    
    \begin{equation}
        G_p(s) = \frac{(s+6)^2}{s(s+1)^2 (s+36)}
    \end{equation}
    
    \begin{center}
        \includegraphics[width=\textwidth]{imagens/cap2_model11_1.eps}
        \captionof{figure}{Processo condicionalmente estável}
        \label{cap2_21}
    \end{center}

\subsection{Processo oscilatório}

    O processo oscilatório é geralmente interessante para o controle \acs{PID}.
    Considerando-se o valor de $zeta$ igual a $0$ ou $1$, e o valor de $\omega _0$
    igual a $1$, $2$, $5$ ou $10$, tipicamente.

    \begin{equation}
        G_p(s) = \frac{\omega _0^2}{(s+1)(s^2+2\zeta \omega _0 s+\omega _0^2)}
    \end{equation}
    
    As figuras abaixo mostram o comportamento do processo para $\zeta$
    igual a 1 e 5, variando-se o valor de $\omega _0$ entre 1, 2 e 10.
    
    \begin{center}
        \includegraphics[width=0.75\textwidth]{imagens/cap2_model12_1.eps}
        \captionof{figure}{Processo oscilatório - $\omega _0 = 1$, $\zeta = 1$}
    \end{center}
    
    \begin{center}
        \includegraphics[width=0.75\textwidth]{imagens/cap2_model12_2.eps}
        \captionof{figure}{Processo oscilatório - $\omega _0 = 1$, $\zeta = 5$}
    \end{center}
    
    \begin{center}
        \includegraphics[width=0.75\textwidth]{imagens/cap2_model12_3.eps}
        \captionof{figure}{Processo oscilatório - $\omega _0 = 2$, $\zeta = 1$}
    \end{center}
    
    \begin{center}
        \includegraphics[width=0.75\textwidth]{imagens/cap2_model12_4.eps}
        \captionof{figure}{Processo oscilatório - $\omega _0 = 2$, $\zeta = 5$}
    \end{center}
    
    \begin{center}
        \includegraphics[width=0.75\textwidth]{imagens/cap2_model12_5.eps}
        \captionof{figure}{Processo oscilatório - $\omega _0 = 10$, $\zeta = 1$}
    \end{center}
    
    \begin{center}
        \includegraphics[width=0.75\textwidth]{imagens/cap2_model12_6.eps}
        \captionof{figure}{Processo oscilatório - $\omega _0 = 10$, $\zeta = 5$}
    \end{center}

\subsection{Processo instável}
    
    O processo instável é dificil de controlar utilizando-se controladores \ac{PID}
    comuns e impossivel de se identificar utilizando-se os métodos por curva de
    reação, pela sua característica de instabilidade.
    
    \begin{equation}
        G_p(s) = \frac{1}{s^2 - 1}
    \end{equation}
    
    \begin{center}
        \includegraphics[width=\textwidth]{imagens/cap2_model13_1.eps}
        \captionof{figure}{Processo instável}
    \end{center}

\subsection{Processo de primeira ordem mais tempo morto com a presença de integrador}
    
    O processo de primeira ordem mais tempo morto com a presença de integrador é
    similar ao processo de primeira ordem mais tempo morto, mas possui um polo
    na origem, dificultando o controle do processo. Vide figura \ref{cap2_26}
    
    \begin{equation}
        G_p(s) = \frac{1}{s(\tau s + 1)}e^{-s}
    \end{equation}
    
    \begin{center}
        \includegraphics[width=\textwidth]{imagens/cap2_model14_1.eps}
        \captionof{figure}{Processo de primeira ordem mais tempo morto com a presença de integrador}
        \label{cap2_26}
    \end{center}

\section{Conclusões parciais}

    Uma quantidade considerável de processos pode ser controlada com as ferramentas
    disponíveis no software produzido por este trabalho, e, mesmo para os processos
    qua não podem ser controlados, a interação com o software proporciona um melhor
    aprendizado e uma melhor compreensão dos conceitos do controlador \ac{PID}
    por parte dos alunos/usuários, proporcionados pela possibilidade de sintonia
    manual do controlado, disponível no software.
