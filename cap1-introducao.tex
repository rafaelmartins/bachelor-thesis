\chapter{Introdução\label{cap:introducao}}

\section*{}

    As teorias de controle de sistemas e os controladores industriais não são um assunto
    novo. Desde o avanço da eletrônica analógia e o surgimento dos amplificadores
    operacionais o homem tenta compensar os erros nos processos industriais de maneira
    automática.

    A habilidade dos controladores \ac{PI} e \ac{PID} para compensar a maioria dos processos
    industriais práticos resultou na larga aceitação dos mesmos nas aplicações industriais
    \cite{Dwyer}. A maioria dos controladores utilizados na industria são do tipo
    \acs{PID} \cite{astrom1984645}. Mais de 90\% das malhas de controle são \acs{PID}
    \cite{astrom20011163}. 

    Desde então os controladores industriais são uma das partes mais importantes de qualquer
    planta industrial, garantindo a estabilidade da operação e fazendo valer as restrições
    do processo.

    Os controladores \acs{PID} são capazes de compensar os 2 tipos de erros encontrados
    nas plantas industriais: o erro de estado transitório, que é o erro encontrado nos momentos
    iniciais da operação da planta, e o erro de estado estacionário, que é o erro de
    ajuste de planta, ou seja, a saida da planta atingiu o nivel de saída esperado, de
    acordo com o ajuste da entrada da planta.

    Os controladores \acs{PID} são constantemente utilizados nas plantas industriais para
    ajustes de nivel, vazão, velocidade e posição, dentre outros, pela sua caracteristica
    de estabilidade e pela facilidade de sintonia, se utilizada uma abordagem prática
    para identificação da planta, que é um dos maiores problemas para os engenheiros.

    São vários os métodos de sintonia para controladores \acs{PID} utilizados em processos
    industriais. Dentre estes, os metodos baseados na resposta ao degrau: Ziegler e Nichols,
    Cohen-Coon e López são os mais utilizados. Estes métodos tradicionais são, até hoje,
    referencia para o estudo dos controladores \ac{PID}, apesar do avanço e das novas
    técnicas disponíveis atualmente.

    Os métodos mais clássicos para a obtenção do modelo dinâmico de um sistema, encontrados
    na literatura, são o método do ganho crítico e a resposta ao degrau, que são capazes
    de aproximar modelos de ordem maior ou igual a 2 de modelos de primeira ordem.

    Este trabalho possui um enfoque didatico, sendo desenvolvido para o uso em sala de aula,
    em aulas de controle. Para isto, o software possui diversas características que estimulam
    o desenvolvimento do aluno e o aprendizado do funcionamento de cada controlador e 
    de cada processo.

    O trabalho também faz intensivo uso de software livre, além de ter originado um software
    livre, como resultado. O uso de software livre na engenharia é relativamente pequeno.
    A maioria das soluções encontradas hoje são proprietárias. A linguagem Python, no entanto,
    vem sendo utilizada no meio científico já a algum tempo \cite{5725235}.


\section{Identicação do problema}
    
    Os controladores industriais evoluiram bastante desde o nascimento, sendo necessária
    a constante atualização do profissional com relação às técnicas de sintonia dos
    controladores e a teoria básica dos mesmos.

    As disciplinas de Controle Automático tem se tornado muito teoricas nos ultimos
    tempos. É dificil para um professor proporcionar experiências práticas para os
    alunos com o material atualmente disponível nas salas de aula das universidades.
    
    Questões concretas ligadas à automação e controle, tais como projetos de
    controladores utilizando as diversas ferramentas analíticas disponíveis, sintonia de
    controladores, compensação dinâmica, aspectos práticos de implementação de sistemas de
    controle, novas técnicas de inteligência computacional aplicadas à automação e controle,
    novas estruturas de controladores, para citar alguns, não são disponibilizados, na estrutura
    curricular normal, para a formação dos alunos \cite{cobenge}.

    Os alunos dos cursos de Engenharia de Controle e Automação precisam ter uma
    visão prática dos problemas encontrados em uma planta industrial, para saberem
    como agir nas situações de problema, portanto torna-se imprescindivel a existencia
    de uma abordagem mais prática dos problemas de engenharia nas disciplinas do curso.

    A dificuldade de uso das ferramentas existentes atualmente é também um limitador
    importante da abordagem prática dos cursos existentes atualmente. Muitas ferramentas
    são dificeis de operar e instalar, ou são caras, ou dependem de sistemas operacionais
    proprietários para funcionar, que também são caros.

\section{Motivação do estudo}

    O curso de Engenharia de Controle e Automação visa formar profissionais
    capazes de lidar com as ferramentas atuais utilizadas nas plantas industriais
    dos mais diversos setores da industria. Os controladores industriais são
    parte importante da maioria das plantas industriais atualmente ativas,
    sendo de fundamental importancia para uma grande gama de processos, portanto
    é extremamente necessário que o engenheiro tenha o conhecimento do funcionamento
    dos controladores, bem como do processo de sintonia dos mesmos.
    
    Atualmente não existem grandes alternativas livres para o estudo e ensino
    de Controladores PID de maneira facil e prática. A maioria das soluções
    disponíveis é construida com base em softwares proprietários de alto custo,
    como o \textbf{MATLAB} $^{\textregistered}$.
    
    Este trabalho nasceu como uma tentativa de entender o funcionamento dos
    controladores PID mais a fundo, e cresceu, originando uma suite completa
    para o estudo de controladores PID, utilizando-se somente software livre,
    aplicado a funções de transferência mais realistas.

    O uso do software livre é um dos grandes diferenciais deste projeto. O projeto
    utiliza-se de ferramentas e bibliotecas prontas, bem como disponibiliza algumas
    novas bibliotecas para a comunidade.

\section{Objetivos do trabalho}

    Este trabalho possui fins didáticos, e foi desenvolvido para proporcionar
    aprendizado tanto a quem desenvolve o software quanto a quem o utiliza em sala
    de aula.
    
    A facilidade de uso e a possibilidade de utilização do software sem instalar
    nada nas maquinas dos alunos também foi levada em consideração durante o
    planejamento do desenvolvimento e a durante a escolha das ferramentas a serem
    utilizadas no projeto.
    
    Um estudante deve ser capaz de operar o software apenas com conhecimentos
    básicos de informática e da teoria dos controladores \acs{PID}, de explorar o
    ambiente e de fazer testes de valores, métodos e processos de maneira facil e
    rápida.
    
    O software é capaz de identificar um processo, a partir de algum dos
    métodos de identificação de sistemas disponíveis (Alfaro \cite{Alfaro2}, Bröida,
    Chen \& Yang \cite{ChenYang}, Ho \cite{Ho}, Smith \cite{Smith} e Vitecková)
    \cite{Viteckova}, sintonizar um controlador \acs{PID},
    utilizando algum dos métodos de sintonia de controladores disponível (Ziegler
    \& Nichols \cite{ZieglerNicholsAsme}, Cohen-Coon \cite{CohenCoon},
    Chien Hrones Reswick 0\%, Chien Hrones Reswick 20\%),
    permitir a sintonia manual do controlador e simular o sistema com o controlador,
    utilizando a sintonia obtida, seja automaticamente ou manualmente.
    
    Um dos grandes objetivos do trabalho era produzir um software de qualidade 
    totalmente baseado em softwares e padrões livres. Desta forma o software,
    além de não possuir custo algum para os alunos e instituições de ensino, permite
    que os alunos participem do desenvolvimento, corrigindo falhas, melhorando
    métodos, adicionando novas funcionalidades, etc.
    
    O uso de ferramentas livres, como um \ac{SCV} aberto, um toolkit para plotagem
    de gráficos aberto e uma linguagem de programação aberta, também foi um objetivo
    importante do trabalho.

\section{Estrutura do trabalho}

    O desenvolvimento deste trabalho está dividida em capítulos conforme sumarizados
    a seguir:

    O Capítulo 2 descreve os processos industriais referenciais utilizados no projeto,
    bem como suas caracteristicas principais.
    
    O Capítulo 3 descreve os metodos de sintonia e identificação de sistemas utilizados
    no projeto e discute as vantagens e desvantagens da escolha de cada um para as
    simulações.
    
    O Capítulo 4 aborda o desenvolvimento do software, discutindo as etapas de
    desenvolvimento e a metodologia utilizada na construção do mesmo.
    
    O Capítulo 5 relata os resultados e as conclusões alcançadas no desenvolvimento do
    trabalho, também são propostas sugestões para trabalhos futuros com objetivo de
    aperfeiçoar o software desenvolvido.
