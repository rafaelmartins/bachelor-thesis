\chapter{Introdução\label{cap:introducao}}

Neste capítulo são apresentados o objetivo deste trabalho e a estrutura básica
do mesmo.

\section{Histórico}

    As teorias de controle de sistemas e os controladores industriais não são um assunto
    novo. Desde o avanço da eletrônica analógia e o surgimento dos amplificadores
    operacionais o homem tenta compensar os erros nos processos industriais de maneira
    automática.

    A habilidade dos controladores \ac{PI} e \ac{PID} para compensar a maioria dos processos
    industriais práticos resultou na larga aceitação dos mesmos nas aplicações industriais
    \cite{Dwyer}.

    Desde então os controladores industriais uma das partes mais importantes de qualquer
    planta industrial, garantindo a estabilidade da operação e fazendo valer as restrições
    do processo.

\section{Identicação do problema}
    
    Os controladores industriais evoluiram bastante desde o nascimento, sendo necessária
    a constante atualização do profissional com relação às técnicas de sintonia dos
    controladores e a teoria básica dos mesmos.

    As disciplinas de Controle Automático tem se tornado muito teoricas nos ultimos
    tempos. É dificil para um professor proporcionar experiências práticas para os
    alunos com o material atualmente disponível nas salas de aula das universidades.

    Os alunos dos cursos de Engenharia de Controle e Automação precisam ter uma
    visão prática dos problemas encontrados em uma planta industrial, para saberem
    como agir nas situações de problema, portanto torna-se imprescindivel a existencia
    de uma abordagem mais prática dos problemas de engenharia nas disciplinas do curso.

    A dificuldade de uso das ferramentas existentes atualmente é também um limitador
    importante da abordagem prática dos cursos existentes atualmente. Muitas ferramentas
    são dificeis de operar e instalar, ou são caras, ou dependem de sistemas operacionais
    proprietários para funcionar, que também são caros.

\section{Motivação do estudo}

    O curso de Engenharia de Controle e Automação visa formar profissionais
    capazes de lidar com as ferramentas atuais utilizadas nas plantas industriais
    dos mais diversos setores da industria. Os controladores industriais são
    parte importante da maioria das plantas industriais atualmente ativas,
    sendo de fundamental importancia para uma grande gama de processos, portanto
    é extremamente necessário que o engenheiro tenha o conhecimento do funcionamento
    dos controladores, bem como do processo de sintonia dos mesmos.
    
    Atualmente não existem grandes alternativas livres para o estudo e ensino
    de Controladores PID de maneira facil e prática. A maioria das soluções
    disponíveis é construida com base em softwares proprietários de alto custo,
    como o \textbf{MATLAB}.
    
    Este trabalho nasceu como uma tentativa de entender o funcionamento dos
    controladores PID mais a fundo, e cresceu, originando uma suite completa
    para o estudo de controladores PID, utilizando-se somente software livre.

    O uso do software livre é um dos grandes diferenciais deste projeto. O projeto
    utiliza-se de ferramentas e bibliotecas prontas, bem como disponibiliza algumas
    novas bibliotecas para a comunidade.

\section{Objetivos do trabalho}

    TODO.

\section{Estrutura do trabalho}

    O desenvolvimento deste trabalho está dividida em capítulos conforme sumarizados a seguir:

    \begin{itemize}
        \item O Capítulo 2 descreve os processos industriais referenciais utilizados no
            projeto, bem como suas caracteristicas principais.
        \item O Capítulo 3 descreve os metodos de sintonia e identificação de sistemas
            utilizados no projeto e discute as vantagens e desvantagens da escolha de
            cada um para as simulações.
        \item O Capítulo 4 aborda o desenvolvimento do software, discutindo as etapas de
            desenvolvimento e a metodologia utilizada na construção do mesmo.
        \item O Capítulo 5 relata os resultados e as conclusões alcançadas no desenvolvimento
            do trabalho, também são propostas sugestões para trabalhos futuros com objetivo
            de aperfeiçoar o software desenvolvido.
    \end{itemize}

