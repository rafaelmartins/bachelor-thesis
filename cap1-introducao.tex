\chapter{Introdução\label{cap:introducao}}

Neste capítulo são apresentados o objetivo desta monografia e a estrutura
da mesma.

\section{Histórico}

    TODO


\section{Motivação do estudo}

    O curso de Engenharia de Controle e Automação visa formar profissionais
    capazes de lidar com as ferramentas atuais utilizadas nas plantas industriais
    dos mais diversos setores da industria. Os controladores industriais são
    parte importante da maioria das plantas industriais atualmente ativas,
    sendo de fundamental importancia para uma grande gama de processos, portanto
    é extremamente necessário que o engenheiro tenha o conhecimento do funcionamento
    dos controladores, bem como do processo de sintonia dos mesmos.
    
    Atualmente não existem grandes alternativas livres para o estudo e ensino
    de Controladores PID de maneira facil e prática. A maioria das soluções
    disponíveis é construida com base em softwares proprietários de alto custo,
    como o \textbf{MATLAB}.
    
    Este trabalho nasceu como uma tentativa de entender o funcionamento dos
    controladores PID mais a fundo, e cresceu, originando uma suite completa
    para o estudo de controladores PID, utilizando-se somente software livre,
    sem ônus algum para os estudantes.


\section{Estrutura da monografia}

    TODO
