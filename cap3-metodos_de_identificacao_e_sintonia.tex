\chapter{Métodos de identificação e sintonia para controladores PID
    baseados no modelo FODT \label{cap:metodos-de-identificacao-e-sintonia}}

Neste capítulo são apresentados os métodos de identificação e sintonia
para controladores PID utilizados no trabalho. Todos os métodos são
baseados no modelo FODT.

\section{Modelo FODT}

    O modelo \ac{FODT}, é um modelo aproximado, bastante utilizado para se
    identificar controladores \acs{PID} em malha aberta. O modelo \acs{FODT}
    possui 3 parametros, o ganho do sistema $K$, a constante de tempo $\tau$
    e o tempo morto $L$, como pode ser visto na equação \ref{eq3_1}.

    \begin{equation}
        G(s) = \frac{K}{1 + s\tau} e^{-sL}
        \label{eq3_1}
    \end{equation}

    Este modelo baseia-se no tempo morto, portanto sistemas sem tempo morto,
    ou com tempo morto muito pequeno, não poderão ser identificados utilizando-se
    este modelo.

    A figura abaixo mostra um exemplo de identificação de processo através
    do modelo \ac{FODT}.

    \begin{center}
        \includegraphics[width=\textwidth]{imagens/cap3_model4_1.eps}
        \captionof{figure}{Modelo FODT - Processo de terceira ordem com tempo morto ajustável}
    \end{center}

\section{Métodos de curva de reação para sintonia de controladores PID}

    O software desenvolvido para este trabalho conta com 4 métodos de
    curva de reação para sintonia de controladores \acs{PID}. Estes métodos
    serão discutidos a seguir.
    
    \subsection{Método de Ziegler-Nichols}
        
        Método de Ziegler-Nichols em malha aberta. Os parâmetros são calculados
        de acordo com a tabela abaixo. Este método se baseia nos 3 parâmetros
        básicos do modelo \acs{FODT}.
        
        \newpage
        
        \begin{center}
            \captionof{table}{Fórmulas para cálculo dos parâmetros - Ziegler-Nichols}
            \begin{tabular}{l*{3}{c}}
Controlador & \multicolumn{3}{c}{Fórmulas} \\
\hline
P   & $Kp = \frac{\tau}{L}$     &              & \\
PI  & $Kp = 0.9 \frac{\tau}{L}$ & $Ti = 3.33L$ & \\
PID & $Kp = 1.2 \frac{\tau}{L}$ & $Ti = 2L$ & $Td = \frac{L}{2}$ \\
            \end{tabular}
        \end{center}
    
    \subsection{Método de Cohen-Coon}
        
        O método de Cohen-Coon também se baseia nos 3 parâmetros básicos
        do modelo \acs{FODT}. Os parâmetros são calculados de acordo com
        a tabela abaixo.
        
        \begin{center}
            \captionof{table}{Fórmulas para cálculo dos parâmetros - Cohen-Coon}
            \begin{tabular}{l*{3}{c}}
Controlador & \multicolumn{3}{c}{Fórmulas} \\
\hline
P   & $Kp = \frac{\tau}{L(1 + \frac{R}{3})}$             &              & \\
PI  & $Kp = \frac{\tau}{L(\frac{10}{9} + \frac{R}{12})}$ & $Ti = L(\frac{30+3R}{9+20R})$ & \\
PD  & $Kp = \frac{\tau}{L(\frac{5}{4} + \frac{R}{6})}$ & & $Td = L(\frac{6-2R}{22+3R})$ \\
PID & $Kp = \frac{\tau}{L(\frac{4}{3} + \frac{R}{4})}$ & $Ti = L(\frac{32+6R}{13+8R})$ & $Td = L(\frac{4}{13+8R})$ \\
            \end{tabular}
        \end{center}
    
    Onde $R = \frac{L}{\tau}$.
    
    \subsection{Método de Chien, Hrones e Reswick}
        
        O método de Chien, Hrones e Reswick é uma modificação do método
        de Ziegler-Nichols, adaptado para que a malha forneça uma resposta ao
        degrau com o menor tempo de subida. Os autores sugeriram um método
        de resposta rápida, sem sobreelevação, ou com 20\% de sobreelevação.
        
        \begin{center}
            \captionof{table}{Fórmulas para cálculo dos parâmetros - Chien, Hrones e Reswick 0\%}
            \begin{tabular}{l*{3}{c}}
Controlador & \multicolumn{3}{c}{Fórmulas} \\
\hline
P   & $Kp = 0.3\frac{\tau}{KL}$     &              & \\
PI  & $Kp = 0.35\frac{\tau}{KL}$ & $Ti = 1.2\tau$ & \\
PID & $Kp = 0.6\frac{\tau}{KL}$ & $Ti = \tau$ & $Td = 0.5L$ \\
            \end{tabular}
        \end{center}

        \newpage

        \begin{center}
            \captionof{table}{Fórmulas para cálculo dos parâmetros - Chien, Hrones e Reswick 20\%}
            \begin{tabular}{l*{3}{c}}
Controlador & \multicolumn{3}{c}{Fórmulas} \\
\hline
P   & $Kp = 0.7\frac{\tau}{KL}$     &              & \\
PI  & $Kp = 0.6\frac{\tau}{KL}$ & $Ti = \tau$ & \\
PID & $Kp = 0.95\frac{\tau}{KL}$ & $Ti = 1.4\tau$ & $Td = 0.47L$ \\
            \end{tabular}
        \end{center}


\section{Métodos heurísticos para identificação de controladores PID}

    TODO.

\section{Conclusões parciais}

    O modelo \acs{FODT}, com seus métodos de identificação e sintonia de
    controladores, é extremamente util, apesar de sua aparente simplicidade e
    facilidade de uso, servindo relativamente bem a diversos casos de uso, onde
    a aplicação de um modelo mais avançado seria altamente custosa.
