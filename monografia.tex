\documentclass[brazil,ruledheader,espaco=umemeio,pagestart=firstchapter]{cefet}

\usepackage[T1]{fontenc}
\usepackage[utf8]{inputenc}
\usepackage[printonlyused]{acronym}
\usepackage[brazil]{babel}
\usepackage{graphicx}
\usepackage{caption}
\usepackage{url}
\usepackage{textcomp}

\begin{document}

\autor{Rafael Gonçalves Martins}
\titulo{PIDSIM - Ferramenta de simulação e controle de processos industriais
    referenciais pela WEB utilizando abordagem FOSS}
\orientador{Marlon José do Carmo}
\comentario{Monografia apresentada para obtenção do Grau de Bacharel em Engenharia de
    Controle e Automação.}
\instituicao{Centro Federal de Educação Tecnológica de Minas Gerais - Campus III}
\local{Leopoldina - Minas Gerais, Brasil}
\data{13 de Julho de 2011}

\capa

\begin{folhadeaprovacao}
    
    \ABNTsign{Prof.: Marlon José do Carmo, Msc. \\ Orientador}
    \ABNTsign{Prof.: Lindopho Oliveira de Araújo Júnior, Dr. Eng.}
    \ABNTsign{Prof.: Luis Claudio Gamboa Lopes, Msc. Eng.}
    \ABNTsign{Prof.: Matusalém Martins Lanes, Msc.}

\end{folhadeaprovacao}

\begin{resumo}

    \textbf{Resumo:} Os controladores \acs{PID} já estão bastante difundidos na industria e no
    meio acadêmico atualmente, mas pouco se conhece a respeito dos métodos
    heurísticos de identificação e sintonia de controladores \acs{PID},
    que por muitas vezes são a única solução viável para o controle de
    uma determinada planta industrial. Sistemas reais geralmente são difíceis
    de modelar matematicamente, devido às complexidades resultantes do
    grande número de elementos presentes nos mesmos. Apresenta-se
    neste trabalho um ambiente multifuncional, desenvolvido em Python,
    plataforma WEB (Internet), para a simulação de controladores \acs{PID},
    utilizando-se catorze processos referenciais disponíveis na literatura, e
    que são capazes de representar uma grande parte dos processos reais
    encontrados no ambiente industrial.
    
    \textbf{Palavras-chave}: Controladores PID, Sintonia, Malhas Industriais,
    Simulação, Educação em Controle.

\end{resumo}

\begin{abstract}

    The \acs{PID} controllers are already widespread in industry and
    academia these days, but little is known about the heuristic methods
    of identification and tuning of \acs{PID} controllers, which often
    are the only viable solution for the control of a given plant. Real
    systems are usually difficult to model mathematically, due to the
    intricacies of the large number of individual components present in
    them. It is presented in this paper a multi-functional environment,
    developed in Python, Web platform (Internet), for the simulation of
    \acs{PID} controllers, using 14 reference models from the literature,
    which are capable of representing a large part of the real processes
    found in the industrial environment.

    KEYWORDS: PID Controllers, Tuning, Industrial Meshes, Simulation,
    Control Education.

\end{abstract}

\chapter*{Agradecimentos}

Aos meus pais, Neusa Maria Gonçalves Martins e Raimundo Regis Gonçalves
Martins, a minha tia Lívia Gonçalves Almeida, às minhas avós Miguelina
(\textit{In memorian}) e Domingas (\textit{In memorian}), pelo amor
incondicional.

Aos meus padrinhos Fátima e Gilberto, pelo apoio de sempre.

Aos amigos da república de Leopoldina, Ulysses, Renan, Helder, João Bernardo
e Pedro, por todos esses anos de convívio.

Ao meu orientador, Marlon José do Carmo, pela confiança.

Aos amigos de turma dos ultimos periodos, Marçal, Scoralick, Daniel,
Alexandre, Murillo, Marcelo, Calil, Ananda, Camila, Matheus, Filipe e
demais. Não foi fácil. :)

A todos os membros do CIERmag (Centro de Imagens e Espectroscopia
\textit{in vivo} por Ressonância Magnética), Prof. Alberto, Mateus, Edson,
Christofer, Mário, Felipe e demais, pela oportunidade de estágio.

Aos amigos de São Carlos, Paulo Matias e Leonardo Amaral, por toda a diversão
nerd dos últimos meses. :)

A toda a comunidade Python, em especial ao criador da linguagem, Guido
van Rossum, pela maravilhosa ferramenta.

A Armin Ronacher pelo \textit{framework} Flask e a John Hunter, Darren
Dale e Michael Droettboom pela biblioteca Matplotlib.

A toda a comunidade de desenvolvedores do Gentoo Linux, em especial ao
amigo Denis Dupeyron, pelo apoio de sempre e pela ajuda com a tradução do
software e com sugestões de melhorias.

A todos aqueles que de alguma forma contribuiram com este projeto, e a
todos aqueles que ainda irão contribuir.

\chapter*{}
\vfill{}
\begin{flushright}
    \emph{Dedico este trabalho a ...}
    %\emph{pode encarar a razão face a face, em}\\
\end{flushright}
\newpage


\listoffigures
\listoftables
\chapter*{Lista de Acrônimos}

\thispagestyle{empty}

\begin{acronym}
    \acro{CEFET-MG}{Centro Federal de Educação Tecnológica de Minas Gerais}
    \acro{PI}{Proporcional Integral}
    \acro{PID}{Proporcional Integral Derivativo}


    \acro{WWW}{World Wide Web}
    \acro{SCV}{Sistema de Controle de Versão}

\end{acronym}

\tableofcontents

\setcounter{page}{1}

\chapter{Introdução\label{cap:introducao}}

Neste capítulo são apresentados o objetivo desta monografia e a estrutura
da mesma.

\section{Histórico}

    TODO


\section{Motivação do estudo}

    O curso de Engenharia de Controle e Automação visa formar profissionais
    capazes de lidar com as ferramentas atuais utilizadas nas plantas industriais
    dos mais diversos setores da industria. Os controladores industriais são
    parte importante da maioria das plantas industriais atualmente ativas,
    sendo de fundamental importancia para uma grande gama de processos, portanto
    é extremamente necessário que o engenheiro tenha o conhecimento do funcionamento
    dos controladores, bem como do processo de sintonia dos mesmos.
    
    Atualmente não existem grandes alternativas livres para o estudo e ensino
    de Controladores PID de maneira facil e prática. A maioria das soluções
    disponíveis é construida com base em softwares proprietários de alto custo,
    como o \textbf{MATLAB}.
    
    Este trabalho nasceu como uma tentativa de entender o funcionamento dos
    controladores PID mais a fundo, e cresceu, originando uma suite completa
    para o estudo de controladores PID, utilizando-se somente software livre,
    sem ônus algum para os estudantes.


\section{Estrutura da monografia}

    TODO

\chapter{Apresentação dos processos referenciais industriais
    \label{cap:processos-referenciais}}

Neste capítulo são apresentados os 14 processos referenciais utilizados
como base para o trabalho.

\section{Processos referenciais industriais}

    TODO.

\section{Conclusões parciais}

    TODO.

\chapter{Métodos de identificação e sintonia para controladores PID
    baseados no modelo FODT \label{cap:fodt}}

Neste capítulo são apresentados os métodos de identificação e sintonia
para controladores PID utilizados no trabalho. Todos os métodos são
baseados no modelo FODT.

\section{Modelo FODT}

    TODO.

\section{Métodos de curva de reação para sintonia de controladores PID}

    TODO.

\section{Métodos heurísticos para sintonia de controladores PID}

    TODO.

\section{Conclusões parciais}

    TODO.

\chapter{O software desenvolvido \label{cap:software}}

Neste capítulo são apresentados aspectos práticos e construtivos do
software desenvolvido.

\section{Introdução ao software}

    Este trabalho se baseia em um software, construido em linguagem Python
    e disponivel via \ac{WWW}. O software, batizado de \textbf{PIDSIM},
    possui os 14 processos industriais referenciais do capítulo
    \ref{cap:processos-referenciais} implementados, bem como diversos
    métodos numéricos para modelagem de sistemas (que serão discutidos no
    Anexo 1), métodos de identificação de sistemas (discutidos no capítulo
    \ref{cap:metodos-de-identificacao-e-sintonia}) e métodos de sintonia de
    controladores \acs{PID} (também discutidos no capítulo
    \ref{cap:metodos-de-identificacao-e-sintonia}).

    O software possui fins didaticos e foi construido visando facilitar a
    utilização por parte dos alunos durante as aulas ou durante o estudo em casa.
    Todas as tecnologias utilizadas são livres e estão disponíveis na internet,
    bem como o código-fonte dos modulos que compõem o software.

    Este software é desenvolvido desde 2009 e passou por grandes modificações
    desde então, até chegar ao resultado disponível hoje.

\section{Arquitetura do software}

    TODO.

\section{A interface WEB}

    TODO.

\section{A linguagem Python}

    TODO.

\section{Conclusões parciais}

    TODO.

\chapter{Conclusões \label{cap:conclusoes}}

Neste capítulo são apresentadas as conclusões e algumas propostas de
trabalhos futuros.

\section{Conclusões}

    TODO.

\section{Trabalhos futuros}

    TODO.


\bibliographystyle{abnt-num}
\bibliography{monografia}

\apendice

\chapter{Métodos numéricos para modelagem de sistemas}

    TODO.

\chapter{Documentação da \acs{API} do software}

    TODO.


\end{document}
