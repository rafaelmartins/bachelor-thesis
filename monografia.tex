\documentclass[brazil,ruledheader,espaco=umemeio,pagestart=firstchapter]{cefet}

\usepackage[T1]{fontenc}
\usepackage[utf8]{inputenc}
\usepackage{acronym}
\usepackage[brazil]{babel}
\usepackage{graphicx}
\usepackage{caption}
\usepackage{url}
\usepackage{textcomp}

\begin{document}

\autor{Rafael Gonçalves Martins}
\titulo{PIDSIM - Ferramenta de simulação e controle de processos industriais
    referenciais pela WEB utilizando abordagem FOSS}
\orientador{Marlon José do Carmo}
\comentario{Monografia apresentada para obtenção do Grau de Bacharel em Engenharia de
    Controle e Automação.}
\instituicao{Centro Federal de Educação Tecnológica de Minas Gerais - Campus III}
\local{Leopoldina - Minas Gerais, Brasil}
\data{13 de Julho de 2011}

\capa

\begin{folhadeaprovacao}
    
    Monografia de Projeto Final de Graduação sob o título
    \textit{``\ABNTtitulodata''}, defendida por \ABNTautordata~e aprovada em
    \ABNTdatadata, em Leopoldina, Estado de Minas Gerais, pela banca examinadora
    constituída pelos professores: \setlength{\ABNTsignthickness}{0.4pt}
    
    \assinatura{Prof.: Marlon José do Carmo, Msc. \\ Orientador}
    \assinatura{Prof.: Lindopho Oliveira Araújo Júnior , Dr. Eng.}
    \assinatura{Prof.: Luis Cláudio Gamboa Lopes, Msc.}
    \assinatura{Prof.: Matusalém Martins Lanes, Msc.}

\end{folhadeaprovacao}

\begin{resumo}

    \textbf{Resumo:} Os controladores \acs{PID} já estão bastante difundidos na industria e no
    meio acadêmico atualmente, mas pouco se conhece a respeito dos métodos
    heurísticos de identificação e sintonia de controladores \acs{PID},
    que por muitas vezes são a única solução viável para o controle de
    uma determinada planta industrial. Sistemas reais geralmente são difíceis
    de modelar matematicamente, devido às complexidades resultantes do
    grande número de elementos presentes nos mesmos. Apresenta-se
    neste trabalho um ambiente multifuncional, desenvolvido em Python,
    plataforma WEB (Internet), para a simulação de controladores \acs{PID},
    utilizando-se catorze processos referenciais disponíveis na literatura, e
    que são capazes de representar uma grande parte dos processos reais
    encontrados no ambiente industrial.
    
    \textbf{Palavras-Chave}: Controladores PID, Sintonia, Malhas Industriais,
    Simulação, Educação em Controle.

\end{resumo}

\begin{abstract}
    
    \begin{center}
        \textbf{PIDSIM - Tool for simulation and control of industrial
        reference models for web with FOSS approach} \\
        \ABNTautordata \\
        2011, July 13
    \end{center}\par{}
    \textbf{Advisor:} \ABNTorientadordata{}
    \vspace*{2.5cm}
    
    \textbf{Abstract:} The \acs{PID} controllers are already widespread in industry and
    academia these days, but little is known about the heuristic methods
    of identification and tuning of \acs{PID} controllers, which often
    are the only viable solution for the control of a given plant. Real
    systems are usually difficult to model mathematically, due to the
    intricacies of the large number of elements present in
    them. It is presented in this paper a multi-functional environment,
    developed in Python, Web platform (Internet), for the simulation of
    \acs{PID} controllers, using fourteen reference models from the literature,
    which are capable of representing a large part of the real processes
    found in the industrial environment.

    \textbf{Keywords:} PID Controllers, Tuning, Industrial Loops, Simulation,
    Control Education.

\end{abstract}

\chapter*{Agradecimentos}

A Deus.

Aos meus pais, Neusa Maria Gonçalves Martins e Raimundo Régis Gonçalves
Martins, a minha tia Lívia Gonçalves Almeida, às minhas avós Miguelina
(\textit{In memorian}) e Domingas (\textit{In memorian}), pelo amor
incondicional.

Aos meus padrinhos Fátima e Gilberto, pelo apoio de sempre.

Aos amigos da república de Leopoldina, Ulysses, Renan, Hélder, João Bernardo
e Pedro, por todos esses anos de convívio.

Ao meu orientador, Marlon José do Carmo, pela confiança.

Aos amigos de turma dos últimos períodos, Marçal, Scoralick, Daniel,
Alexandre, Murillo, Marcelo, Calil, Ananda, Camila, Matheus, Filipe e
demais. Não foi fácil. :)

A todos os membros do CIERmag (Centro de Imagens e Espectroscopia
\textit{in vivo} por Ressonância Magnética), Prof. Alberto, Mateus, Edson,
Christofer, Mário, Felipe e demais, pela oportunidade de estágio. Ao Paulo
Matias e ao Leonardo Amaral, pela ajuda em São Carlos.

A toda a comunidade Python, em especial ao criador da linguagem, Guido
van Rossum, pela maravilhosa ferramenta.

A Armin Ronacher pelo \textit{framework} Flask e a John Hunter, Darren
Dale e Michael Droettboom pela biblioteca Matplotlib.

A toda a comunidade de desenvolvedores do Gentoo Linux, em especial ao
amigo Denis Dupeyron, pelo apoio de sempre e pela ajuda com a tradução do
software e com sugestões de melhorias.

A todos aqueles que de alguma forma contribuíram com este projeto, e a
todos aqueles que ainda irão contribuir.

\chapter*{}
\vfill{}
\begin{flushright}
    \emph{Dedico este trabalho a ...}
    %\emph{pode encarar a razão face a face, em}\\
\end{flushright}
\newpage


\listoffigures
\listoftables
\chapter*{Lista de Acrônimos}

\begin{acronym}
    \acro{CEFET-MG}{Centro Federal de Educação Tecnológica de Minas Gerais}
\end{acronym}

\tableofcontents

\setcounter{page}{1}

\chapter{Introdução\label{cap:introducao}}

\section*{}

    As teorias de controle de sistemas e os controladores industriais não são um assunto
    novo. Desde o avanço da eletrônica analógia e o surgimento dos amplificadores
    operacionais o homem tenta compensar os erros nos processos industriais de maneira
    automática.

    A habilidade dos controladores \ac{PI} e \ac{PID} para compensar a maioria dos processos
    industriais práticos resultou na larga aceitação dos mesmos nas aplicações industriais
    \cite{Dwyer}. A maioria dos controladores utilizados na industria são do tipo
    \acs{PID} \cite{astrom1984645}. Mais de 90\% das malhas de controle são \acs{PID}
    \cite{astrom20011163}. 

    Desde então os controladores industriais são uma das partes mais importantes de qualquer
    planta industrial, garantindo a estabilidade da operação e fazendo valer as restrições
    do processo.

    Os controladores \acs{PID} são capazes de compensar os 2 tipos de erros encontrados
    nas plantas industriais: o erro de estado transitório, que é o erro encontrado nos momentos
    iniciais da operação da planta, e o erro de estado estacionário, que é o erro de
    ajuste de planta, ou seja, a saida da planta atingiu o nivel de saída esperado, de
    acordo com o ajuste da entrada da planta.

    Os controladores \acs{PID} são constantemente utilizados nas plantas industriais para
    ajustes de nivel, vazão, velocidade e posição, dentre outros, pela sua caracteristica
    de estabilidade e pela facilidade de sintonia, se utilizada uma abordagem prática
    para identificação da planta, que é um dos maiores problemas para os engenheiros.

    São vários os métodos de sintonia para controladores \acs{PID} utilizados em processos
    industriais. Dentre estes, os metodos baseados na resposta ao degrau: Ziegler e Nichols,
    Cohen-Coon e López são os mais utilizados. Estes métodos tradicionais são, até hoje,
    referencia para o estudo dos controladores \ac{PID}, apesar do avanço e das novas
    técnicas disponíveis atualmente.

    Os métodos mais clássicos para a obtenção do modelo dinâmico de um sistema, encontrados
    na literatura, são o método do ganho crítico e a resposta ao degrau, que são capazes
    de aproximar modelos de ordem maior ou igual a 2 de modelos de primeira ordem.

    Este trabalho possui um enfoque didatico, sendo desenvolvido para o uso em sala de aula,
    em aulas de controle. Para isto, o software possui diversas características que estimulam
    o desenvolvimento do aluno e o aprendizado do funcionamento de cada controlador e 
    de cada processo.

    O trabalho também faz intensivo uso de software livre, além de ter originado um software
    livre, como resultado. O uso de software livre na engenharia é relativamente pequeno.
    A maioria das soluções encontradas hoje são proprietárias. A linguagem Python, no entanto,
    vem sendo utilizada no meio científico já a algum tempo \cite{5725235}.


\section{Identicação do problema}
    
    Os controladores industriais evoluiram bastante desde o nascimento, sendo necessária
    a constante atualização do profissional com relação às técnicas de sintonia dos
    controladores e a teoria básica dos mesmos.

    As disciplinas de Controle Automático tem se tornado muito teoricas nos ultimos
    tempos. É dificil para um professor proporcionar experiências práticas para os
    alunos com o material atualmente disponível nas salas de aula das universidades.
    
    Questões concretas ligadas à automação e controle, tais como projetos de
    controladores utilizando as diversas ferramentas analíticas disponíveis, sintonia de
    controladores, compensação dinâmica, aspectos práticos de implementação de sistemas de
    controle, novas técnicas de inteligência computacional aplicadas à automação e controle,
    novas estruturas de controladores, para citar alguns, não são disponibilizados, na estrutura
    curricular normal, para a formação dos alunos \cite{cobenge}.

    Os alunos dos cursos de Engenharia de Controle e Automação precisam ter uma
    visão prática dos problemas encontrados em uma planta industrial, para saberem
    como agir nas situações de problema, portanto torna-se imprescindivel a existencia
    de uma abordagem mais prática dos problemas de engenharia nas disciplinas do curso.

    A dificuldade de uso das ferramentas existentes atualmente é também um limitador
    importante da abordagem prática dos cursos existentes atualmente. Muitas ferramentas
    são dificeis de operar e instalar, ou são caras, ou dependem de sistemas operacionais
    proprietários para funcionar, que também são caros.

\section{Motivação do estudo}

    O curso de Engenharia de Controle e Automação visa formar profissionais
    capazes de lidar com as ferramentas atuais utilizadas nas plantas industriais
    dos mais diversos setores da industria. Os controladores industriais são
    parte importante da maioria das plantas industriais atualmente ativas,
    sendo de fundamental importancia para uma grande gama de processos, portanto
    é extremamente necessário que o engenheiro tenha o conhecimento do funcionamento
    dos controladores, bem como do processo de sintonia dos mesmos.
    
    Atualmente não existem grandes alternativas livres para o estudo e ensino
    de Controladores PID de maneira facil e prática. A maioria das soluções
    disponíveis é construida com base em softwares proprietários de alto custo,
    como o \textbf{MATLAB} $^{\textregistered}$.
    
    Este trabalho nasceu como uma tentativa de entender o funcionamento dos
    controladores PID mais a fundo, e cresceu, originando uma suite completa
    para o estudo de controladores PID, utilizando-se somente software livre,
    aplicado a funções de transferência mais realistas.

    O uso do software livre é um dos grandes diferenciais deste projeto. O projeto
    utiliza-se de ferramentas e bibliotecas prontas, bem como disponibiliza algumas
    novas bibliotecas para a comunidade.

\section{Objetivos do trabalho}

    Este trabalho possui fins didáticos, e foi desenvolvido para proporcionar
    aprendizado tanto a quem desenvolve o software quanto a quem o utiliza em sala
    de aula.
    
    A facilidade de uso e a possibilidade de utilização do software sem instalar
    nada nas maquinas dos alunos também foi levada em consideração durante o
    planejamento do desenvolvimento e a durante a escolha das ferramentas a serem
    utilizadas no projeto.
    
    Um estudante deve ser capaz de operar o software apenas com conhecimentos
    básicos de informática e da teoria dos controladores \acs{PID}, de explorar o
    ambiente e de fazer testes de valores, métodos e processos de maneira facil e
    rápida.
    
    O software é capaz de identificar um processo, a partir de algum dos
    métodos de identificação de sistemas disponíveis (Alfaro \cite{Alfaro2}, Bröida,
    Chen \& Yang \cite{ChenYang}, Ho \cite{Ho}, Smith \cite{Smith} e Vitecková)
    \cite{Viteckova}, sintonizar um controlador \acs{PID},
    utilizando algum dos métodos de sintonia de controladores disponível (Ziegler
    \& Nichols \cite{ZieglerNicholsAsme}, Cohen-Coon \cite{CohenCoon},
    Chien Hrones Reswick 0\%, Chien Hrones Reswick 20\%),
    permitir a sintonia manual do controlador e simular o sistema com o controlador,
    utilizando a sintonia obtida, seja automaticamente ou manualmente.
    
    Um dos grandes objetivos do trabalho era produzir um software de qualidade 
    totalmente baseado em softwares e padrões livres. Desta forma o software,
    além de não possuir custo algum para os alunos e instituições de ensino, permite
    que os alunos participem do desenvolvimento, corrigindo falhas, melhorando
    métodos, adicionando novas funcionalidades, etc.
    
    O uso de ferramentas livres, como um \ac{SCV} aberto, um toolkit para plotagem
    de gráficos aberto e uma linguagem de programação aberta, também foi um objetivo
    importante do trabalho.

\section{Estrutura do trabalho}

    O desenvolvimento deste trabalho está dividida em capítulos conforme sumarizados
    a seguir:

    O Capítulo 2 descreve os processos industriais referenciais utilizados no projeto,
    bem como suas caracteristicas principais.
    
    O Capítulo 3 descreve os metodos de sintonia e identificação de sistemas utilizados
    no projeto e discute as vantagens e desvantagens da escolha de cada um para as
    simulações.
    
    O Capítulo 4 aborda o desenvolvimento do software, discutindo as etapas de
    desenvolvimento e a metodologia utilizada na construção do mesmo.
    
    O Capítulo 5 relata os resultados e as conclusões alcançadas no desenvolvimento do
    trabalho, também são propostas sugestões para trabalhos futuros com objetivo de
    aperfeiçoar o software desenvolvido.

\chapter{Os processos referenciais industriais
    \label{cap:processos-referenciais}}

\section{Introdução}

Neste capítulo são apresentados os 14 processos referenciais utilizados
como base para este trabalho. Estes processos referenciais são capazes de se
aproximar de uma grande quantidade dos processos reais encontrados na indústria,
sendo interessantes para o estudo e até mesmo para a sintonia dos controladores
encontrados nas plantas industriais. Estes processos já foram amplamente destacados
na literatura \cite{Isermann} \cite{AstromHagglund}.

Vale ressaltar que, como se trata de um projeto didático, nem todos os processos
conseguem ser identificados e controlados utilizando-se os métodos propostos
neste trabalho, e que a análise da eficiência dos métodos faz parte do aprendizado
do aluno que utiliza-se deste software como ferramenta de estudo dos controladores
\acs{PID}.

\section{Processos referenciais industriais}

\subsection{Processo de primeira ordem}
    
    O processo de primeira ordem é o mais simples de todo, sendo bastante encontrado
    nas plantas industriais. É capaz de representar uma boa quantidade de motores,
    entre outros processos.
    
    Devido às suas caracteristicas dinâmicas (definidas pelo ganho estático $k$ e
    pela constante de tempo $\tau$), pelo seu curto tempo de subida, este processo
    é dificil de se identificar através de métodos que utilizem a curva de reação.
    
    \begin{equation}
        G_p(s) = \frac{k}{(1+\tau s)}
    \end{equation}
    
    \begin{center}
        \includegraphics[width=\textwidth]{imagens/cap2_model1_1.eps}
        \captionof{figure}{Processo de primeira ordem}
    \end{center}

\subsection{Processo de segunda ordem}
    
    O processo de segunda ordem é também bastante difundido na industria. Possui
    características dinâmicas definidas pelo ganho estático $k$ e os tempos $T_1$
    e $T_2$.
    
    Por não apresentar um crescimento monotônico, este processo é mais facil de ser
    identificado a partir dos métodos de identificação utilizando a curva de reação
    do que o processo referencial de primeira ordem.
    
    \begin{equation}
        G_p(s) = \frac{k}{(1+T_1 s)(1+T_2 s)}
    \end{equation}
    
    \begin{center}
        \includegraphics[width=\textwidth]{imagens/cap2_model2_1.eps}
        \captionof{figure}{Processo de segunda ordem}
    \end{center}

\subsection{Processo de segunda ordem de fase não-mínima}

    O processo de segunda ordem de fase não-minima tem uma dinâmica diferente dos
    outros processos já apresentados. Este processo apresenta resposta inversa à
    entrada, quando recebendo um sinal em forma de degrau, até um certo limite,
    passando a rastrear o sinal de entrada (referência).
    
    Não apresenta grandes dificuldades para a identificação utilizando os métodos
    de identificação por curva de reação, dependendo dos parâmetros do processo,
    que são o ganho estático $k$ e os tempos $T_1$ e $T_2$.

    \begin{equation}
        G_p(s) = \frac{k(1-T_1 s)}{(1+T_1 s)(1+T_2 s)}
    \end{equation}

    \begin{center}
        \includegraphics[width=\textwidth]{imagens/cap2_model3_1.eps}
        \captionof{figure}{Processo de segunda ordem de fase não-mínima}
    \end{center}

\subsection{Processo de terceira ordem com tempo morto ajustável}
    
    O processo de terceira ordem pode assumir várias dinâmicas diferentes, dependendo
    dos parâmetros do sistema, que são o ganho estático $k$, os tempos $T_1$, $T_2$,
    $T_3$ e $T_4$ e o tempo morto $T_t$.
    
    Processos com tempo morto tendem a ser mais faceis de identificar utilizando-se
    os metodos de identificação através da curva de reação, porém são mais dificeis
    de se controlar com controladores \acs{PID} convencionais.
    
    \begin{equation}
        G_p(s) = \frac{k(1+T_4 s)}{(1+T_1 s)(1+T_2 s)(1+T_3 s)} e^{-T_t s}
    \end{equation}
    
    \begin{center}
        \includegraphics[width=\textwidth]{imagens/cap2_model4_1.eps}
        \captionof{figure}{Processo de terceira ordem com tempo morto ajustável}
    \end{center}

\subsection{Processo de pólos múltiplos e iguais}

    O processo de pólos múltiplos e iguais possui dinâmica variável, dependendo
    do parâmetro $n$, que representa a ordem do sistema.
    
    O aumento da ordem do sistema tende a facilitar a identificação através dos
    métodos de identificação por curva de reação.
    
    Os valores de $n$ variam de 1 a 4, tipicamente.
    
    \begin{equation}
        G_p(s) = \frac{1}{(s+1)^n}
    \end{equation}

    \begin{center}
        \includegraphics[width=0.75\textwidth]{imagens/cap2_model5_1.eps}
        \captionof{figure}{Processo de pólos múltiplos e iguais - $n = 1$}
    \end{center}
    
    \begin{center}
        \includegraphics[width=0.75\textwidth]{imagens/cap2_model5_2.eps}
        \captionof{figure}{Processo de pólos múltiplos e iguais - $n = 2$}
    \end{center}
    
    \begin{center}
        \includegraphics[width=0.75\textwidth]{imagens/cap2_model5_3.eps}
        \captionof{figure}{Processo de pólos múltiplos e iguais - $n = 3$}
    \end{center}
    
    \begin{center}
        \includegraphics[width=0.75\textwidth]{imagens/cap2_model5_4.eps}
        \captionof{figure}{Processo de pólos múltiplos e iguais - $n = 4$}
    \end{center}

\subsection{Processo de quarta ordem}
    
    O processo de quarta ordem tem dinâmica definida pelo parametro $\alpha$,
    sendo facilmente controlavel para valores pequenos de $\alpha$, entre 0 e 1.
    
    Para $\alpha = 1$, este processo se comporta como o anterior, para $n = 4$.
    
    \begin{equation}
        G_p(s) = \frac{1}{(s+1)(\alpha s+1)(\alpha ^2 s+1)(\alpha ^3 s+1)}
    \end{equation}

    \begin{center}
        \includegraphics[width=0.75\textwidth]{imagens/cap2_model6_1.eps}
        \captionof{figure}{Processo de quarta ordem - $\alpha = 0.2$}
    \end{center}
    
    \begin{center}
        \includegraphics[width=0.75\textwidth]{imagens/cap2_model6_2.eps}
        \captionof{figure}{Processo de quarta ordem - $\alpha = 0.5$}
    \end{center}

\subsection{Processo com três pólos iguais e zero no semi-plano direito}

    O processo com três pólos iguais e zero no semi-plano direito do plano $s$
    possui dinâmica definida pelo parametro $\alpha$. Quanto maior o valor de
    $\alpha$, maior a dificuldade para se controlar o processo utilizando-se
    um controlador \acs{PID} convencional.
    
    Por possuir tempo morto, pode ser considerado um processo facil de identificar
    utilizando-se os métodos de identificação por curva de reação.

    \begin{equation}
        G_p(s) = \frac{1-\alpha s}{(s+1)^3}
    \end{equation}

    \begin{center}
        \includegraphics[width=0.75\textwidth]{imagens/cap2_model7_1.eps}
        \captionof{figure}{Processo com três pólos iguais e zero no semi-plano direito - $\alpha = 0.2$}
    \end{center}

    \begin{center}
        \includegraphics[width=0.75\textwidth]{imagens/cap2_model7_2.eps}
        \captionof{figure}{Processo com três pólos iguais e zero no semi-plano direito - $\alpha = 0.5$}
    \end{center}
    
    \begin{center}
        \includegraphics[width=0.75\textwidth]{imagens/cap2_model7_3.eps}
        \captionof{figure}{Processo com três pólos iguais e zero no semi-plano direito - $\alpha = 1$}
    \end{center}
    
    \begin{center}
        \includegraphics[width=0.75\textwidth]{imagens/cap2_model7_4.eps}
        \captionof{figure}{Processo com três pólos iguais e zero no semi-plano direito - $\alpha = 5$}
    \end{center}

\subsection{Processo de primeira ordem com tempo morto}

    O processo de primeira ordem com tempo morto é considerado um processo
    clássico, e é bastante utilizado no estudo de controladores \acs{PID}. O
    processo possui dinâmica definida pela constante de tempo $\tau$ e pelo
    tempo morto $L$. Neste trabalho definimos o valor do tempo morto $L$ como 1.
    
    Assim como a maioria dos processos com tempo morto, pode ser considerado
    fácil de identificar utilizando métodos de curva de reação e complicado para
    controlar utilizando controladores \acs{PID} comuns.

    \begin{equation}
        G_p(s) = \frac{1}{(\tau s +1)}e^{-s}
    \end{equation}
    
    Nos gráficos abaixo existe oscilação no tempo morto, devido à aproximação de
    padè utilizada no software. Os graficos foram gerados para uma aproximação
    de quinta ordem.

    \begin{center}
        \includegraphics[width=0.75\textwidth]{imagens/cap2_model8_1.eps}
        \captionof{figure}{Processo de primeira ordem com tempo morto - $\tau = 0.5$}
    \end{center}

    \begin{center}
        \includegraphics[width=0.75\textwidth]{imagens/cap2_model8_2.eps}
        \captionof{figure}{Processo de primeira ordem com tempo morto - $\tau = 2$}
    \end{center}
    
    \begin{center}
        \includegraphics[width=0.75\textwidth]{imagens/cap2_model8_3.eps}
        \captionof{figure}{Processo de primeira ordem com tempo morto - $\tau = 10$}
    \end{center}

\subsection{Processo de segunda ordem com tempo morto}
    
    O Processo de segunda ordem com tempo morto é similar ao de primeira ordem
    com tempo morto, com diferenças de dinâmica dadas pela diferença de ordem
    dos processos e pela constante de tempo $\tau$, que possui os seguintes valores,
    típicamente: $0; 0,1; 0,2; 0,5; 2; 5; 10$.
    
    \begin{equation}
        G_p(s) = \frac{1}{(\tau s +1)^2}e^{-s}
    \end{equation}

    \begin{center}
        \includegraphics[width=0.75\textwidth]{imagens/cap2_model9_1.eps}
        \captionof{figure}{Processo de segunda ordem com tempo morto - $\tau = 2$}
    \end{center}
    
    \begin{center}
        \includegraphics[width=0.75\textwidth]{imagens/cap2_model9_2.eps}
        \captionof{figure}{Processo de segunda ordem com tempo morto - $\tau = 5$}
    \end{center}

\subsection{Processo com características dinâmicas assimétricas}
    
    O processo com características dinâmicas assimétricas possui duas características
    dinâmicas distintas, sendo uma rápida, com ganho estático igual a 1 e constante
    de tempo igual a 1, e outra lenta, com um ganho estático igual a 10.
    
    Este processo é dificil de controlar utilizando se controladores \acs{PID}
    simples e identificando o processo com metodos de curva de reação, por conta das
    características assimétricas do processo.
    
    \begin{equation}
        G_p(s) = \frac{100}{(s+10)^2}\left ( \frac{1}{s+1} + \frac{0,5}{s+0,05} \right )
    \end{equation}
    
    \begin{center}
        \includegraphics[width=\textwidth]{imagens/cap2_model10_1.eps}
        \captionof{figure}{Processo com características dinâmicas assimétricas}
    \end{center}

\subsection{Processo condicionalmente estável}
    
    O processo condicionalmente estável pode se comportar como um processo estável
    ou instável, dependendo do ajuste do ponto de operação dos controladores.
    
    \begin{equation}
        G_p(s) = \frac{(s+6)^2}{s(s+1)^2 (s+36)}
    \end{equation}
    
    \begin{center}
        \includegraphics[width=\textwidth]{imagens/cap2_model11_1.eps}
        \captionof{figure}{Processo condicionalmente estável}
    \end{center}

\subsection{Processo oscilatório}

    O processo oscilatório é geralmente interessante para o controle \acs{PID}.
    Considerando-se o valor de $zeta$ igual a $0$ ou $1$, e o valor de $\omega _0$
    igual a $1$, $2$, $5$ ou $10$, tipicamente.

    \begin{equation}
        G_p(s) = \frac{\omega _0^2}{(s+1)(s^2+2\zeta \omega _0 s+\omega _0^2)}
    \end{equation}
    
    As figuras abaixo mostram o comportamento do processo para $\zeta = 1$,
    variando-se o valor de $\omega _0$:
    
    \begin{center}
        \includegraphics[width=0.75\textwidth]{imagens/cap2_model12_1.eps}
        \captionof{figure}{Processo oscilatório - $\omega _0 = 1$}
    \end{center}
    
    \begin{center}
        \includegraphics[width=0.75\textwidth]{imagens/cap2_model12_2.eps}
        \captionof{figure}{Processo oscilatório - $\omega _0 = 2$}
    \end{center}
    
    \begin{center}
        \includegraphics[width=0.75\textwidth]{imagens/cap2_model12_3.eps}
        \captionof{figure}{Processo oscilatório - $\omega _0 = 10$}
    \end{center}

\subsection{Processo instável}
    
    O processo instável é dificil de controlar utilizando-se controladores \ac{PID}
    comuns e até mesmo de se identificar utilizando-se os métodos por curva de
    reação, pela sua característica de instabilidade.
    
    \begin{equation}
        G_p(s) = \frac{1}{s^2 - 1}
    \end{equation}
    
    \begin{center}
        \includegraphics[width=\textwidth]{imagens/cap2_model13_1.eps}
        \captionof{figure}{Processo instável}
    \end{center}

\subsection{Processo de primeira ordem mais tempo morto com a presença de integrador}
    
    O processo de primeira ordem mais tempo morto com a presença de integrador é
    similar ao processo de primeira ordem mais tempo morto, mas possui um polo
    na origem, dificultando o controle do processo.
    
    \begin{equation}
        G_p(s) = \frac{1}{s(\tau s + 1)}e^{-s}
    \end{equation}
    
    \begin{center}
        \includegraphics[width=\textwidth]{imagens/cap2_model14_1.eps}
        \captionof{figure}{Processo de primeira ordem mais tempo morto com a presença de integrador}
    \end{center}

\section{Conclusões parciais}

    Uma quantidade considerável de processos pode ser controlada com as ferramentas
    disponíveis no software produzido por este trabalho, e, mesmo para os processos
    qua não podem ser controlados, a interação com o software proporciona um melhor
    aprendizado e uma melhor compreensão dos conceitos do controlador \ac{PID}
    por parte dos alunos/usuários.

\chapter{Métodos de identificação e sintonia para controladores PID
    baseados no modelo FODT \label{cap:metodos-de-identificacao-e-sintonia}}

Neste capítulo são apresentados os métodos de identificação e sintonia
para controladores PID utilizados no trabalho. Todos os métodos são
baseados no modelo FODT.

\section{Modelo FODT}

    O modelo \ac{FODT}, é um modelo aproximado, bastante utilizado para se
    identificar controladores \acs{PID} em malha aberta. O modelo \acs{FODT}
    possui 3 parâmetros, o ganho do sistema $K$, a constante de tempo $\tau$
    e o tempo morto $L$, como pode ser visto na equação \ref{eq3_1}.

    \begin{equation}
        G(s) = \frac{K}{1 + s\tau} e^{-sL}
        \label{eq3_1}
    \end{equation}

    Este modelo baseia-se no tempo morto, portanto sistemas sem tempo morto,
    ou com tempo morto muito pequeno, não poderão ser identificados utilizando-se
    este modelo.

    A figura abaixo mostra um exemplo de identificação de processo através
    do modelo \ac{FODT}.

    \begin{center}
        \includegraphics[width=\textwidth]{imagens/cap3_model4_1.eps}
        \captionof{figure}{Modelo FODT - Processo de terceira ordem com tempo morto ajustável}
    \end{center}

\section{Métodos de curva de reação para sintonia de controladores PID}

    O software desenvolvido para este trabalho conta com 4 métodos de
    curva de reação para sintonia de controladores \acs{PID}. Estes métodos
    serão discutidos a seguir.
    
    \subsection{Método de Ziegler-Nichols}
        
        Método de Ziegler-Nichols em malha aberta. Os parâmetros são calculados
        de acordo com a tabela abaixo. Este método se baseia nos 3 parâmetros
        básicos do modelo \acs{FODT}.
        
        \newpage
        
        \begin{center}
            \captionof{table}{Fórmulas para cálculo dos parâmetros - Ziegler-Nichols}
            \begin{tabular}{l*{3}{c}}
Controlador & \multicolumn{3}{c}{Fórmulas} \\
\hline
P   & $Kp = \frac{\tau}{L}$     &              & \\
PI  & $Kp = 0.9 \frac{\tau}{L}$ & $Ti = 3.33L$ & \\
PID & $Kp = 1.2 \frac{\tau}{L}$ & $Ti = 2L$ & $Td = \frac{L}{2}$ \\
            \end{tabular}
        \end{center}
    
    \subsection{Método de Cohen-Coon}
        
        O método de Cohen-Coon também se baseia nos 3 parâmetros básicos
        do modelo \acs{FODT}. Os parâmetros são calculados de acordo com
        a tabela abaixo.
        
        \begin{center}
            \captionof{table}{Fórmulas para cálculo dos parâmetros - Cohen-Coon}
            \begin{tabular}{l*{3}{c}}
Controlador & \multicolumn{3}{c}{Fórmulas} \\
\hline
P   & $Kp = \frac{\tau}{L(1 + \frac{R}{3})}$             &              & \\
PI  & $Kp = \frac{\tau}{L(\frac{10}{9} + \frac{R}{12})}$ & $Ti = L(\frac{30+3R}{9+20R})$ & \\
PD  & $Kp = \frac{\tau}{L(\frac{5}{4} + \frac{R}{6})}$ & & $Td = L(\frac{6-2R}{22+3R})$ \\
PID & $Kp = \frac{\tau}{L(\frac{4}{3} + \frac{R}{4})}$ & $Ti = L(\frac{32+6R}{13+8R})$ & $Td = L(\frac{4}{13+8R})$ \\
            \end{tabular}
        \end{center}
    
    Onde $R = \frac{L}{\tau}$.
    
    \subsection{Método de Chien, Hrones e Reswick}
        
        O método de Chien, Hrones e Reswick é uma modificação do método
        de Ziegler-Nichols, adaptado para que a malha forneça uma resposta ao
        degrau com o menor tempo de subida. Os autores sugeriram um método
        de resposta rápida, sem sobrelevação, ou com 20\% de sobrelevação.
        
        \begin{center}
            \captionof{table}{Fórmulas para cálculo dos parâmetros - Chien, Hrones e Reswick 0\%}
            \begin{tabular}{l*{3}{c}}
Controlador & \multicolumn{3}{c}{Fórmulas} \\
\hline
P   & $Kp = 0.3\frac{\tau}{KL}$     &              & \\
PI  & $Kp = 0.35\frac{\tau}{KL}$ & $Ti = 1.2\tau$ & \\
PID & $Kp = 0.6\frac{\tau}{KL}$ & $Ti = \tau$ & $Td = 0.5L$ \\
            \end{tabular}
        \end{center}

        \newpage

        \begin{center}
            \captionof{table}{Fórmulas para cálculo dos parâmetros - Chien, Hrones e Reswick 20\%}
            \begin{tabular}{l*{3}{c}}
Controlador & \multicolumn{3}{c}{Fórmulas} \\
\hline
P   & $Kp = 0.7\frac{\tau}{KL}$     &              & \\
PI  & $Kp = 0.6\frac{\tau}{KL}$ & $Ti = \tau$ & \\
PID & $Kp = 0.95\frac{\tau}{KL}$ & $Ti = 1.4\tau$ & $Td = 0.47L$ \\
            \end{tabular}
        \end{center}


\section{Métodos heurísticos para identificação de controladores PID}

    Os métodos heurísticos para identificação de controladores \acs{PID}
    utilizando-se o modelo \acs{FODT} são geralmente baseados em 2 pontos,
    atrelados ao local onde a curva do gráfico da resposta ao degrau em
    malha aberta alcança certo nível em relação à amplitude máxima da curva
    ou ao ponto de estabilização do sistema.
    
    Os dois pontos do método de identificação definem uma reta, conhecida
    como reta de sintonia. A distancia do ponto onde a reta de sintonia
    toca o eixo do tempo até sua origem é o tempo morto $L$. O valor de $\tau$
    é a constante de tempo, medida da projeção do extremo da reta de sintonia
    alinhado com o ponto máximo da curva até o ponto onde a reta de sintonia
    toca o eixo dos tempos.
    
    Segue uma tabela com os pontos utilizados por cada método de identificação.
    
    \begin{center}
        \captionof{table}{Constantes para os métodos de identificação}
        \begin{tabular}{l l l}
Método & $\%p1(t1)$ & $\%p2(t2)$ \\
\hline
Alfaro                    & 25.0 & 75.0 \\
Broida                    & 28.0 & 40.0 \\
Chen \& Yang              & 33.0 & 67.0 \\
Ho \textit{et al.}        & 35.0 & 85.0 \\
Smith                     & 28.3 & 63.2 \\
Vitecková \textit{et al.} & 33.0 & 70.0 \\
        \end{tabular}
    \end{center}
    
    No software do projeto, os pontos e a reta são definidos geometricamente,
    de acordo com a tabela.    

\section{Conclusões parciais}

    O modelo \acs{FODT}, com seus métodos de identificação e sintonia de
    controladores, é extremamente útil, apesar de sua aparente simplicidade e
    facilidade de uso, servindo relativamente bem a diversos casos de uso, onde
    a aplicação de um modelo mais avançado seria altamente custosa.

\chapter{O software desenvolvido \label{cap:software}}

Neste capítulo são apresentados aspectos práticos e construtivos do
software desenvolvido.

\section{Introdução ao software}

    Este trabalho se baseia em um software, construido em linguagem Python
    e disponivel via \ac{WWW}. O software, batizado de \textbf{PIDSIM},
    possui os 14 processos industriais referenciais do capítulo
    \ref{cap:processos-referenciais} implementados, bem como diversos
    métodos numéricos para modelagem de sistemas (que serão discutidos no
    Anexo 1), métodos de identificação de sistemas (discutidos no capítulo
    \ref{cap:metodos-de-identificacao-e-sintonia}) e métodos de sintonia de
    controladores \acs{PID} (também discutidos no capítulo
    \ref{cap:metodos-de-identificacao-e-sintonia}).

    O software possui fins didaticos e foi construido visando facilitar a
    utilização por parte dos alunos durante as aulas ou durante o estudo em casa.
    Todas as tecnologias utilizadas são livres e estão disponíveis na internet,
    bem como o código-fonte dos modulos que compõem o software.

    Este software é desenvolvido desde 2009 e passou por grandes modificações
    desde então, até chegar ao resultado disponível hoje.

\section{Arquitetura do software}

    TODO.

\section{A interface WEB}

    TODO.

\section{A linguagem Python}

    TODO.

\section{Conclusões parciais}

    TODO.

\chapter{Conclusões \label{cap:conclusoes}}

Neste capítulo são apresentadas as conclusões e algumas propostas de
trabalhos futuros.

\section{Conclusões}

    O software apresentado neste trabalho é capaz de auxiliar estudantes
    durante o curso de disciplinas de controle, mostrando não somente
    as situações onde os métodos implementados funcionam, mas também as
    falhas, forçando o aluno a descobrir as causas e, consequentemente, a
    entender melhor o funcionamento dos controladores \acs{PID}.

    O PIDSIM é um software simples, fácil de utilizar e com uma arquitetura
    bem definida, capaz de absorver contribuições nas mais diversas áreas,
    seja uma nova interface gráfica, novos métodos de identificação
    e sintonia ou traduções para outros idiomas além do inglês e do português.
    
    A metodologia de desenvolvimento de software livre, combinada ao uso
    de um sistema de controle de versão descentralizado (o Mercurial), vem
    contribuir para que haja um desenvolvimento mais rápido e colaborativo,
    envolvendo todas as pessoas interessadas com facilidade.
    
    Porém, mesmo com o crescimento constante do software livre nos últimos tempos,
    ainda é pequena a quantidade de softwares disponíveis para engenharia
    e suas subáreas, apesar já existem grandes ferramentas disponíveis,
    que tornam possível o desenvolvimento de novas aplicações de maneira
    confortável, sugerindo que poderão nascer novas aplicações em breve,
    

\section{Trabalhos futuros}

    A arquitetura modular do software permite uma enorme expansão, o que
    resulta em uma grande quantidade de possíveis trabalhos futuros.
    Segue uma breve lista de necessidades imediatas do projeto:
    
    \begin{itemize}
        \item Adição de novos métodos de identificação de sistemas, alem dos
            métodos do modelo \acs{FODT}
        \item Adição de novos métodos de sintonia de controladores, alem dos
            métodos do modelo \acs{FODT}
        \item Criação de uma interface para Desktop.
        \item Implementação do pacote pidsim.core em linguagem C, para melhorar
            o desempenho do software.
    \end{itemize}


\bibliographystyle{abnt-num}
\bibliography{monografia}

\apendice

\chapter{Métodos numéricos para modelagem de sistemas}

    Métodos numéricos são essenciais para a discretização e modelagem
    dos sistemas estudados neste trabalho.
    
    De posse de um modelo em Espaço de Estados do sistema a ser discretizado,
    alguns métodos largamente discutidos na literatura podem ser utilizados
    para se obter o modelo discretizado, e posteriormente realizar a plotagem
    do sistema.
    
    Os métodos utilizados no trabalho são:
    
    \begin{itemize}
        \item Euler
        \item Runge-Kutta de $2^a$ ordem
        \item Runge-Kutta de $3^a$ ordem
        \item Runge-Kutta de $4^a$ ordem
    \end{itemize}
    

\section{Código-fonte dos métodos numéricos}
    
    \footnotesize
    \begin{verbatim}
# -*- coding: utf-8 -*-
"""
    pidsim.core.discretization
    ~~~~~~~~~~~~~~~~~~~~~~~~~~

    Transfer Function discretization methods.
    
    This module implements some numerical methods to discretize the
    Transfer Functions on the time domain.
    
    :copyright: 2009-2010 by Rafael Goncalves Martins
    :license: GPL-2, see LICENSE for more details.
"""

#TODO: discretize State-Space models too.
#TODO: implement more numerical methods

__all__ = ['Euler', 'RungeKutta2', 'RungeKutta3', 'RungeKutta4']

from pidsim.core.error import ControlSystemsError
from pidsim.core.types import Matrix, ZerosMatrix, IdentityMatrix, \
    TransferFunction, StateSpace


def Euler(g, sample_time, total_time):
    """Euler Method
    
    Returns the points of the step response of the transfer function 'g',
    discretized with the Euler method, using the sample time 'sample_time'
    on 'total_time' seconds. For example::
    
        >>> g = TransferFunction([1], [1, 2, 3])
        >>> t, y = Euler(g, 0.01, 10)
        >>> print t
        (prints a vector of times 0-10s, with the sample time 0.01s)
        >>> print y
        (prints a vector of points, with the same size of 't')
    
    """
    
    if not isinstance(g, TransferFunction):
        raise ControlSystemsError('Parameter must be a Transfer Fcn.')

    ss = StateSpace(g)
    
    samples = int(total_time/sample_time)
    
    t = [sample_time * a for a in range(samples+1)]
    
    x = ZerosMatrix(ss.a.rows, 1)
    y = [0.0]
    
    eye = IdentityMatrix(ss.a.rows)
    
    a1 = eye + ss.a.mult(sample_time)
    a2 = ss.b.mult(sample_time)
    
    for i in range(samples):
        x = a1*x + a2
        y.append((ss.c*x)[0][0] + ss.d[0][0])

    return t, y


def RungeKutta2(g, sample_time, total_time):
    """RungeKutta2 Method
    
    Returns the points of the step response to the transfer function 'g',
    discretized with the Runge Kutta (order 2) method, using the sample
    time 'sample_time' on 'total_time' seconds. For example::
    
        >>> g = TransferFunction([1], [1, 2, 3])
        >>> t, y = RungeKutta2(g, 0.01, 10)
        >>> print t
        (prints a vector of times 0-10s, with the sample time 0.01s)
        >>> print y
        (prints a vector of points, with the same size of 't')
    
    """
    
    if not isinstance(g, TransferFunction):
        raise ControlSystemsError('Parameter must be a Transfer Fcn.')

    ss = StateSpace(g)
    
    samples = int(total_time/sample_time)
    
    t = [sample_time * a for a in range(samples+1)]
    
    x = ZerosMatrix(ss.a.rows, 1)
    y = [0.0]
    
    eye = IdentityMatrix(ss.a.rows)
    
    a1 = ss.a * ss.a
    a2 = ss.a.mult(2) + a1.mult(sample_time)
    a3 = a2.mult(0.5)
    a4 = ss.a.mult(sample_time)
    a5 = a4*ss.b + ss.b.mult(2)
    a6 = eye + a3.mult(sample_time)
    a7 = a5.mult(sample_time/2)
    
    for i in range(samples):
        x = a6*x + a7
        y.append((ss.c*x)[0][0] + ss.d[0][0])

    return t, y


def RungeKutta3(g, sample_time, total_time):
    """RungeKutta3 Method
    
    Returns the points of the step response to the transfer function 'g',
    discretized with the Runge Kutta (order 3) method, using the sample
    time 'sample_time' on 'total_time' seconds. For example::
    
        >>> g = TransferFunction([1], [1, 2, 3])
        >>> t, y = RungeKutta3(g, 0.01, 10)
        >>> print t
        (prints a vector of times 0-10s, with the sample time 0.01s)
        >>> print y
        (prints a vector of points, with the same size of 't')
    
    """
    
    if not isinstance(g, TransferFunction):
        raise ControlSystemsError('Parameter must be a Transfer Fcn.')

    ss = StateSpace(g)
    
    samples = int(total_time/sample_time)
    
    t = [sample_time * a for a in range(samples+1)]
    
    x = ZerosMatrix(ss.a.rows, 1)
    y = [0.0]
    
    eye = IdentityMatrix(ss.a.rows)
    
    a1 = ss.a.mult(sample_time) # A*T
    a2 = ss.b.mult(sample_time) # B*T
    a3 = (ss.a * ss.a).mult(sample_time * sample_time) # A^2*T^2
    a4 = a3.mult(0.5) # (A^2*T^2)/2
    a5 = (ss.a * ss.b).mult(sample_time * sample_time) # A*B*T^2
    a6 = a5.mult(0.5) # (A*B*T^2)/2
    a7 = a3.mult(3.0/4.0) # (A^2*T^2)*(3/4)
    a8 = (ss.a*ss.a*ss.a).mult(sample_time * sample_time * sample_time)
    a9 = a8.mult(3.0/8.0)
    a10 = (ss.a*ss.a*ss.b).mult(sample_time * sample_time * sample_time)
    a11 = a10.mult(3.0/8.0)
    a12 = a5.mult(3.0/4.0) + a2
    
    for i in range(samples):
        k1 = a1*x + a2
        k2 = (a1 + a4)*x + a6 + a2
        k3 = (a1 + a7 + a9)*x + a11 + a12
        
        x = x + (k1.mult(2.0/9.0) + k2.mult(1.0/3.0) + k3.mult(4.0/9.0))
        
        y.append((ss.c*x)[0][0] + ss.d[0][0])

    return t, y


def RungeKutta4(g, sample_time, total_time):
    """RungeKutta4 Method
    
    Returns the points of the step response to the transfer function 'g',
    discretized with the Runge Kutta (order 4) method, using the sample
    time 'sample_time' on 'total_time' seconds. For example::
    
        >>> g = TransferFunction([1], [1, 2, 3])
        >>> t, y = RungeKutta4(g, 0.01, 10)
        >>> print t
        (prints a vector of times 0-10s, with the sample time 0.01s)
        >>> print y
        (prints a vector of points, with the same size of 't')
    
    """
    
    if not isinstance(g, TransferFunction):
        raise ControlSystemsError('Parameter must be a Transfer Fcn.')

    ss = StateSpace(g)
    
    samples = int(total_time/sample_time)
    
    t = [sample_time * a for a in range(samples+1)]
    
    x = ZerosMatrix(ss.a.rows, 1)
    y = [0.0]
    
    eye = IdentityMatrix(ss.a.rows)
    
    a1 = ss.a.mult(sample_time) # A*T
    a2 = ss.b.mult(sample_time) # B*T
    a3 = (ss.a * ss.a).mult(sample_time * sample_time) # A^2*T^2
    a4 = a3.mult(0.5) # (A^2*T^2)/2
    a5 = (ss.a * ss.b).mult(sample_time * sample_time) # A*B*T^2
    a6 = a5.mult(0.5) # (A*B*T^2)/2
    a7 = (ss.a*ss.a*ss.a).mult(sample_time * sample_time * sample_time)
    a8 = a7.mult(0.25)
    a9 = (ss.a*ss.a*ss.b).mult(sample_time * sample_time * sample_time)
    a10 = a9.mult(0.25)
    a11 = a7.mult(0.5)
    a12 = (ss.a*ss.a*ss.a*ss.a).mult(sample_time * sample_time * \
           sample_time * sample_time)
    a13 = a12.mult(0.25)
    a14 = ss.a.mult(sample_time * sample_time)
    a15 = a14*ss.b
    a16 = a9.mult(0.5)
    a17 = (ss.a*ss.a*ss.a*ss.b).mult(sample_time * sample_time * \
           sample_time * sample_time)
    a18 = a17.mult(0.25)
    
    for i in range(samples):
        k1 = a1*x + a2
        k2 = (a1 + a4)*x + a6 + a2
        k3 = (a1 + a4 + a8)*x + a2 + a6 + a10;
        k4 = (a1 + a3 + a11 + a13)*x + a15 +a16 + a18 + a2
        
        x = x + k1.mult(1.0/6.0) + k2.mult(1.0/3.0) + k3.mult(1.0/3.0) \
              + k4.mult(1.0/6.0)
        
        y.append((ss.c*x)[0][0] + ss.d[0][0])

    return t, y\end{verbatim}
    \normalsize

\chapter{Documentação da \acs{API} do software}

    Este capítulo mostra a documentação da \ac{API} dos 3 pacotes de
    software desenvolvidos para o projeto.


\section{pidsim.core}

    Pacote principal do software, com toda a implementação básica das 
    funcionalidades do software.

\subsection{pidsim.core.types.Polynomial}

    \begin{verbatim}
Polynomial(args)
    \end{verbatim}

    Classe que representa o tipo de dados Polinômio, baseado em listas
    do Python. Suporta soma, subtração, multiplicação (por outro polinômio
    ou por um numero inteiro) e divisão.
    
    Exemplo de uso:

    \begin{verbatim}
>>> a = Polynomial([1, 2, 3])
>>> b = Polynomial([2, 3, 4])
>>> print a + b
3x^2 + 5x + 7
    \end{verbatim}
    
\subsection{pidsim.core.types.Matrix}

    \begin{verbatim}
Matrix(args)
    \end{verbatim}

    Classe que implementa o tipo de dados Matriz, baseado em listas do
    Python. O objeto matriz é uma lista de listas e posssui 2 propriedades:
    \textit{cols} (número de colunas) e \textit{rows} (número de linhas).
    
    Suporta soma, subtração, multiplicação (por outra matriz ou por um
    número inteiro), transposição e acesso direto a linhas e colunas.
    
    Exemplo de uso:
    
    \begin{verbatim}
>>> a = Matrix([
...     [1, 2, 3],
...     [2, 3, 4],
...     [3, 4, 5],
... ])
>>>
>>> print a
1    2    3
2    3    4
3    4    5
>>>
>>> a.rows
3
>>> a.cols
3
>>> print a(1)
2    3    4
>>> print a(1, 1)
3
    \end{verbatim}

\subsection{pidsim.core.types.TransferFunction}

    \begin{verbatim}
TransferFunction(num, den)
    \end{verbatim}

    Classe que implementa o tipo de dados Função de Transferencia, baseado
    no tipo de dados Polinômio. O objeto função de transferencia posssui
    2 propriedades: \textit{num} (numerador) e \textit{den} (denominador).
    
    Suporta soma, subtração, multiplicação (por outra função ou por um
    número inteiro), divisão por um número inteiro, simplificação e
    \textit{feedback} com ganho unitário.
    
    Exemplo de uso:
    
    \begin{verbatim}
>>> a = TransferFunction([1], [1, 2, 3])
>>> b = TransferFunction([1], [2, 3, 4])
>>> c = a * b
>>> print c
Transfer Function:

              1               
------------------------------
2s^4 + 7s^3 + 16s^2 + 17s + 12
    \end{verbatim}

\subsection{pidsim.core.types.StateSpace}

    \begin{verbatim}
StateSpace(a_or_tf, b=None, c=None, d=None)
    \end{verbatim}

    Classe que implementa o tipo de dados Espaço de Estados, baseado
    nos tipos de dados Função de Transferencia e Matriz. O objeto função
    de transferencia posssui 4 propriedades: \textit{a}, \textit{b},
    \textit{c} e \textit{d}.
    
    Se o primeiro argumento for uma função de transferência, será transformado
    automaticamente em um modelo em espaço de estados durante a criação do
    objeto. Funcionalidade baseada na função \textit{tf2ss} do GNU Octave
    (careçe de citação) que é parte do \textit{Octave Control Systems Toolbox}.
    
    Exemplo de uso:
    
    \begin{verbatim}
>>> a = TransferFunction([1], [1,2,3])
>>> b = StateSpace(a)
>>> print b
State-Space model:

Matrix A:
0   1   
-3.0    -2.0    

Matrix B:
0   
1   

Matrix C:
1.0 0.0 

Matrix D:
0.0 
    \end{verbatim}

\subsection{pidsim.core.discretization.Euler}

    \begin{verbatim}
t, y = Euler(g, sample_time, total_time)
    \end{verbatim}

    Função que implementa a discretização de funções de transferência,
    utilizando o método de Euler.
    
    Recebe uma função de transferência, o tempo de amostragem e o tempo
    total.

    Retorna uma tupla de 2 vetores, que são o vetor dos tempos e das amplitudes,
    respectivamente.

\subsection{pidsim.core.discretization.RungeKutta2}

    \begin{verbatim}
t, y = RungeKutta2(g, sample_time, total_time)
    \end{verbatim}
    
    Função que implementa a discretização de funções de transferência,
    utilizando o método de RungeKutta de $2^a$ ordem.
    
    Recebe uma função de transferência, o tempo de amostragem e o tempo
    total.
    
    Retorna uma tupla de 2 vetores, que são o vetor dos tempos e das amplitudes,
    respectivamente.

\subsection{pidsim.core.discretization.RungeKutta3}

    \begin{verbatim}
t, y = RungeKutta3(g, sample_time, total_time)
    \end{verbatim}
    
    Função que implementa a discretização de funções de transferência,
    utilizando o método de RungeKutta de $3^a$.
    
    Recebe uma função de transferência, o tempo de amostragem e o tempo
    total.
    
    Retorna uma tupla de 2 vetores, que são o vetor dos tempos e das amplitudes,
    respectivamente.

\subsection{pidsim.core.discretization.RungeKutta4}

    \begin{verbatim}
t, y = RungeKutta4(g, sample_time, total_time)
    \end{verbatim}
    
    Função que implementa a discretização de funções de transferência,
    utilizando o método de RungeKutta de $4^a$.
    
    Recebe uma função de transferência, o tempo de amostragem e o tempo
    total.
    
    Retorna uma tupla de 2 vetores, que são o vetor dos tempos e das amplitudes,
    respectivamente.

\subsection{pidsim.core.error.ControlSystemsError}

    \begin{verbatim}
raise ControlSystemsError(error_message)
    \end{verbatim}
    
    Função que implementa a exceção básica que é provocada pelo software.
    
    Herda todos os parametros e atributos da exceção padrão do Python.

\subsection{pidsim.core.helpers.get\_time\_near}

    \begin{verbatim}
t = get_time_near(t, y, point)
    \end{verbatim}
    
    Função que retorna o tempo '$t$' do ponto mais próximo ao ponto
    desejado '$point$', presente nos vetores '$t$' (tempos) e '$y$'
    (amplitudes).
    
\subsection{pidsim.core.pade.Pade\{1,5\}}

    \begin{verbatim}
g = Pade1(t)
g = Pade2(t)
g = Pade3(t)
g = Pade4(t)
g = Pade5(t)
    \end{verbatim}
    
    Funções que implementam a aproximação de Padè, para $1^a$, $2^a$, $3^a$,
    $4^a$ e $5^a$ ordens.
    
    Recebem um tempo de atraso de transporte desejado e retornam uma
    função de transferencia, para ser multiplicada com a função de transferencia
    original.


\end{document}
