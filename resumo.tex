\begin{resumo}

    Os controladores \acs{PID} já estão bastante difundidos na industria e no
    meio acadêmico atualmente, mas pouco se conhece a respeito dos métodos
    heurísticos de identificação e sintonia de controladores \acs{PID},
    que por muitas vezes são a unica solução viável para o controle de
    uma determinada planta industrial. Sistemas reais geralmente são dificeis
    de modelar matemáticamente, devido às complexidades resultantes do
    grande número de componentes individuais presentes nos mesmos. Apresenta-se
    neste trabalho um ambiente multifuncional, desenvolvido em Python,
    plataforma WEB (Internet), para a simulação de controladores \acs{PID},
    utilizando-se 14 processos referênciais disponíveis na literatura, e
    que são capazes de representar uma grande parte dos processos reais
    encontrados no ambiente industrial.

    PALAVRAS-CHAVE: Controladores PID, Sintonia, Malhas Industriais,
    Simulação, Educação em Controle.

\end{resumo}
