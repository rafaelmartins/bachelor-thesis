\chapter{O software desenvolvido \label{cap:software}}

Neste capítulo são apresentados aspectos práticos e construtivos do
software desenvolvido.

\section{Introdução ao software}

    Este trabalho se baseia em um software, construido em linguagem Python
    e disponivel via \ac{WWW}. O software, batizado de \textbf{PIDSIM},
    possui os 14 processos industriais referenciais do capítulo
    \ref{cap:processos-referenciais} implementados, bem como diversos
    métodos numéricos para modelagem de sistemas (que serão discutidos no
    Anexo 1), métodos de identificação de sistemas (discutidos no capítulo
    \ref{cap:metodos-de-identificacao-e-sintonia}) e métodos de sintonia de
    controladores \acs{PID} (também discutidos no capítulo
    \ref{cap:metodos-de-identificacao-e-sintonia}).

    O software possui fins didaticos e foi construido visando facilitar a
    utilização por parte dos alunos durante as aulas ou durante o estudo em casa.
    Todas as tecnologias utilizadas são livres e estão disponíveis na internet,
    bem como o código-fonte dos modulos que compõem o software.

    Este software é desenvolvido desde 2009 e passou por grandes modificações
    desde então, até chegar ao resultado disponível hoje.

\section{Arquitetura do software}

    TODO.

\section{A interface WEB}

    TODO.

\section{A linguagem Python}

    TODO.

\section{Conclusões parciais}

    TODO.
