\chapter{O software desenvolvido \label{cap:software}}

Neste capítulo são apresentados aspectos práticos e construtivos do
software desenvolvido.

\section{Introdução}

    Este trabalho se baseia em um software, construido em linguagem Python
    e disponivel via \ac{WWW}. O software, batizado de \textbf{PIDSIM},
    possui os 14 processos industriais referenciais apresentados no capítulo
    \ref{cap:processos-referenciais} implementados, bem como diversos
    métodos numéricos para modelagem de sistemas (que serão discutidos no
    Anexo 1), métodos de identificação de sistemas (discutidos no capítulo
    \ref{cap:metodos-de-identificacao-e-sintonia}) e métodos de sintonia de
    controladores \acs{PID} (também discutidos no capítulo
    \ref{cap:metodos-de-identificacao-e-sintonia}).

    O software possui fins didáticos e foi construido visando facilitar a
    utilização por parte dos alunos durante as aulas ou durante o estudo em casa.
    Todas as tecnologias utilizadas são livres e estão disponíveis na internet,
    bem como o código-fonte dos modulos que compõem o software.

    Este software é desenvolvido desde 2009 e passou por grandes modificações
    desde então, até chegar ao resultado disponível hoje.

    Há uma instância pública do software rodando no seguinte endereço: \\
    \url{http://pidsim.rafaelmartins.eng.br}

    O código-fonte do software está disponível em repositórios \textbf{Mercurial}: \\
    \url{http://hg.rafaelmartins.eng.br/pidsim/}

    O software está licenciado sob a licença \ac{GPL-2}, podendo ser redistribuido
    livremente, de acordo com os termos da licença.

\section{Arquitetura do software}

    Uma das características marcantes do software livre é a possibilidade de
    usuários trabalharem no código-fonte do software e adicionar as funcionalidades
    que desejar, porém isto se torna dificil se a arquitetura do software não tiver
    sido bem planejada.

    Levando se a facilidade de expansão em consideração, foi construida uma estrutura
    modular para o software, dividindo-o em pacotes individuais, e com o minimo de
    interdependência entre si.
    
    O software foi dividido em 3 pacotes Python:

    \begin{itemize}
        \item pidsim.core
        \item pidsim.models
        \item pidsim.web
    \end{itemize}

    Cada pacote tem sua funcionalidade especifica definida, e pode ser desenvolvido
    separadamente, facilitando bastante o desenvolvimento distribuido do software,
    que é uma das metas deste projeto.

    A adição e remoção de métodos é extremamente simples, facilitando as contribuições
    dos estudantes.

    Segue uma breve descrição de cada pacote Python que compõe o software:

    \subsection{pidsim.core}

        O pacote \textbf{pidsim.core} é o pacote principal do software, sendo responsável
        por toda a implementação básica, como os tipos de dados (com suporte a conversão
        de tipos e sobrecarga de operadores), os métodos numéricos, os algorítmos de controles,
        entre outras funcionalidades centrais do software.

        Este pacote foi implementado em Python puro, visando a possibilidade de utilização
        em sistemas onde não fosse possível compilar modulos Python escritos em linguagem C.
        Além disso, por se tratar de um projeto didático, não seria justificavel a utilização
        de um grande numero de bibliotecas prontas.

        \subsubsection{Tipos básicos de dados}

            Foram implementados os seguintes tipos básicos de dados:
        
            \begin{itemize}
                \item Polinômio
                \item Matriz
                \item Matriz Identidade
                \item Função de Transferência
                \item Espaço de Estados
            \end{itemize}

            Um ponto importante desta implementação de tipos de dados é a conversão de Função
            de Transferência para Espaco de Estados, que é vital para a discretização dos
            processos.

        \subsubsection{Métodos de discretização de processos}

            Foram implementados os seguintes métodos de discretização de processos:

            \begin{itemize}
                \item Euler
                \item RungeKutta de $2^a$ ordem
                \item RungeKutta de $3^a$ ordem
                \item RungeKutta de $4^a$ ordem
            \end{itemize}

            O estudante pode facilmente escolher qual dos métodos utilizar.

        \subsubsection{Aproximação de Padè}
            
            A implementação da aproximação de Padè criada é capaz de simular o tempo morto
            para alguns sistemas.

            Foi implementada a aproximação de Padè de $1^a$, $2^a$, $3^a$, $4^a$ e $5^a$ ordem.

        \subsubsection{Métodos de identificação de sistemas utilizando-se a curva de reação}
            
            Foram implementados os seguintes métodos de identificação dos sistemas utilizando-se
            a curva de reação:

            \begin{itemize}
                \item Alfaro
                \item Bröida
                \item Chen \& Yang
                \item Ho \textit{et al.}
                \item Smith
                \item Vitecková \textit{et al.}
            \end{itemize}

            Cada método possui suas características particulares, que já foram discutidas no capítulo
            \ref{cap:metodos-de-identificacao-e-sintonia}.

        \subsubsection{Métodos de sintonia para controladores \acs{PID}}
            
            Foram implementados os seguintes métodos de sintonia para controladores \acs{PID}:

            \begin{itemize}
                \item Ziegler \& Nichols
                \item Cohen \& Coon
                \item Chien \& Hrones \& Reswick 0\% de \textit{Overshoot}
                \item Chien \& Hrones \& Reswick 20\% de \textit{Overshoot}
            \end{itemize}

            Estes métodos também já foram discutidos no capítulo
            \ref{cap:metodos-de-identificacao-e-sintonia}.

        \subsection{pidsim.models}
            
            O pacote \textbf{pidsim.models} possui implementados os 14 processos referenciais
            discutidos no capítulo \ref{cap:processos-referenciais}.

            Cada processo é uma classe Python, com atributos específicos, como a expressão
            \LaTeX que representa o processo, um método Python que representa a função de
            transferencia básica do processo, uma lista de argumentos que esta função recebe,
            entre outros objetos necessários ao uso dos processos pela interface Web, ou qualquer
            outra interface que venha a ser desenvolvida.

        \subsection{pidsim.web}
            
            O pacote \textbf{pidsim.web} implementa a interface web, desenvolvida utilizando-se
            o \textit{framework web} \textbf{Flask} e o toolkit de plotagem de gráficos \textbf{Matplotlib}.

            Apesar da facilidade em se implementar interfaces para o software, devido à arquitetura
            implementada, apenas a interface web está disponível no momento.

\section{A interface WEB}

    A interface do software com o usuario é feita através da Internet, utilizando-se um aplicativo Web,
    desenvolvido com o \textit{framework web} \textbf{Flask}.

    Uma imagem com a tela inicial do software pode ser vista abaixo:
    
    \begin{center}
        \includegraphics[width=0.6\textwidth]{imagens/cap4-pidsim_home.eps}
        \captionof{figure}{Tela inicial do software}
    \end{center}

    O software possui suporte a internacionalização. Utilizando se o pacote Python Babel. Atualmente está
    implementado o suporte ao Inglês e ao Português.

\section{A linguagem Python}

    TODO.

\section{Conclusões parciais}

    TODO.
