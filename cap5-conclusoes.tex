\chapter{Conclusões \label{cap:conclusoes}}

Neste capítulo são apresentadas as conclusões e algumas propostas de
trabalhos futuros.

\section{Conclusões}

    O software apresentado neste trabalho é capaz de auxiliar estudantes
    durante o curso de disciplinas de controle, mostrando não somente
    as situações onde os métodos implementados funcionam, mas também as
    falhas, forçando o aluno a descobrir as causas e, consequentemente, a
    entender melhor o funcionamento dos controladores \acs{PID}.

    O PIDSIM é um software simples, facil de utilizar e com uma arquitetura
    bem definida, capaz de absorver contribuições nas mais diversas areas,
    seja uma nova interface gráfica, novos métodos de identificação
    e sintonia ou traduções para outros idiomas além do inglês e do português.
    
    A metodologia de desenvolvimento de software livre, combinada ao uso
    de um sistema de controle de versão descentralizado (o Mercurial), vem
    contribuir para que haja um desenvolvimento mais rápido e colaborativo,
    envolvendo todas as pessoas interessadas com facilidade.
    
    Porém, mesmo com o crescimento constante do software livre nos ultimos tempos,
    ainda é pequena a quantidade de softwares disponíveis para engenharia
    e suas sub-areas, apesar já existem grandes ferramentas disponíveis,
    que tornam possível o desenvolvimento de novas aplicações de maneira
    confortavel, sugerindo que poderão nascer novas aplicações em breve,
    

\section{Trabalhos futuros}

    A arquitetura modular do software permite uma enorme expansão, o que
    resulta em uma grande quantidade de possiveis trabalhos futuros.
    Segue uma breve lista de necessidades imediatas do projeto:
    
    \begin{itemize}
        \item Adição de novos métodos de identificação de sistemas, alem dos
            métodos do modelo \acs{FODT}
        \item Adição de novos métodos de sintonia de controladores, alem dos
            métodos do modelo \acs{FODT}
        \item Criação de uma interface para Desktop.
        \item Implementação do pacote pidsim.core em linguagem C, para melhorar
            o desempenho do software.
    \end{itemize}
