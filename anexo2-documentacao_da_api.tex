\chapter{Documentação da \acs{API} do software}

    Este capítulo mostra a documentação da \ac{API} dos 3 pacotes de
    software desenvolvidos para o projeto.


\section{pidsim.core}

    Pacote principal do software, com toda a implementação básica das 
    funcionalidades do software.

\subsection{pidsim.core.types.Polynomial}

    \begin{verbatim}
Polynomial(args)\end{verbatim}

    Classe que representa o tipo de dados Polinômio, baseado em listas
    do Python. Suporta soma, subtração, multiplicação (por outro polinômio
    ou por um numero inteiro) e divisão.
    
    Exemplo de uso:

    \begin{verbatim}
>>> a = Polynomial([1, 2, 3])
>>> b = Polynomial([2, 3, 4])
>>> print a + b
3x^2 + 5x + 7\end{verbatim}
    
\subsection{pidsim.core.types.Matrix}

    \begin{verbatim}
Matrix(args)\end{verbatim}

    Classe que implementa o tipo de dados Matriz, baseado em listas do
    Python. O objeto matriz é uma lista de listas e posssui 2 propriedades:
    \textit{cols} (número de colunas) e \textit{rows} (número de linhas).
    
    Suporta soma, subtração, multiplicação (por outra matriz ou por um
    número inteiro), transposição e acesso direto a linhas e colunas.
    
    Exemplo de uso:
    
    \begin{verbatim}
>>> a = Matrix([
...     [1, 2, 3],
...     [2, 3, 4],
...     [3, 4, 5],
... ])
>>>
>>> print a
1    2    3
2    3    4
3    4    5
>>>
>>> a.rows
3
>>> a.cols
3
>>> print a(1)
2    3    4
>>> print a(1, 1)
3\end{verbatim}

\subsection{pidsim.core.types.TransferFunction}

    \begin{verbatim}
TransferFunction(num, den)\end{verbatim}

    Classe que implementa o tipo de dados Função de Transferencia, baseado
    no tipo de dados Polinômio. O objeto função de transferencia posssui
    2 propriedades: \textit{num} (numerador) e \textit{den} (denominador).
    
    Suporta soma, subtração, multiplicação (por outra função ou por um
    número inteiro), divisão por um número inteiro, simplificação e
    \textit{feedback} com ganho unitário.
    
    Exemplo de uso:
    
    \begin{verbatim}
>>> a = TransferFunction([1], [1, 2, 3])
>>> b = TransferFunction([1], [2, 3, 4])
>>> c = a * b
>>> print c
Transfer Function:

              1               
------------------------------
2s^4 + 7s^3 + 16s^2 + 17s + 12\end{verbatim}

\subsection{pidsim.core.types.StateSpace}

    \begin{verbatim}
StateSpace(a_or_tf, b=None, c=None, d=None)
    \end{verbatim}

    Classe que implementa o tipo de dados Espaço de Estados, baseado
    nos tipos de dados Função de Transferencia e Matriz. O objeto função
    de transferencia posssui 4 propriedades: \textit{a}, \textit{b},
    \textit{c} e \textit{d}.
    
    Se o primeiro argumento for uma função de transferência, será transformado
    automaticamente em um modelo em espaço de estados durante a criação do
    objeto. Funcionalidade baseada na função \textit{tf2ss} do GNU Octave
    (careçe de citação) que é parte do \textit{Octave Control Systems Toolbox}.
    
    Exemplo de uso:
    
    \begin{verbatim}
>>> a = TransferFunction([1], [1,2,3])
>>> b = StateSpace(a)
>>> print b
State-Space model:

Matrix A:
0   1   
-3.0    -2.0    

Matrix B:
0   
1   

Matrix C:
1.0 0.0 

Matrix D:
0.0\end{verbatim}

\subsection{pidsim.core.discretization.Euler}

    \begin{verbatim}
t, y = Euler(g, sample_time, total_time)\end{verbatim}

    Função que implementa a discretização de funções de transferência,
    utilizando o método de Euler.
    
    Recebe uma função de transferência, o tempo de amostragem e o tempo
    total.

    Retorna uma tupla de 2 vetores, que são o vetor dos tempos e das amplitudes,
    respectivamente.

\subsection{pidsim.core.discretization.RungeKutta2}

    \begin{verbatim}
t, y = RungeKutta2(g, sample_time, total_time)\end{verbatim}
    
    Função que implementa a discretização de funções de transferência,
    utilizando o método de RungeKutta de $2^a$ ordem.
    
    Recebe uma função de transferência, o tempo de amostragem e o tempo
    total.
    
    Retorna uma tupla de 2 vetores, que são o vetor dos tempos e das amplitudes,
    respectivamente.

\subsection{pidsim.core.discretization.RungeKutta3}

    \begin{verbatim}
t, y = RungeKutta3(g, sample_time, total_time)\end{verbatim}
    
    Função que implementa a discretização de funções de transferência,
    utilizando o método de RungeKutta de $3^a$ ordem.
    
    Recebe uma função de transferência, o tempo de amostragem e o tempo
    total.
    
    Retorna uma tupla de 2 vetores, que são o vetor dos tempos e das amplitudes,
    respectivamente.

\subsection{pidsim.core.discretization.RungeKutta4}

    \begin{verbatim}
t, y = RungeKutta4(g, sample_time, total_time)\end{verbatim}
    
    Função que implementa a discretização de funções de transferência,
    utilizando o método de RungeKutta de $4^a$ ordem.
    
    Recebe uma função de transferência, o tempo de amostragem e o tempo
    total.
    
    Retorna uma tupla de 2 vetores, que são o vetor dos tempos e das amplitudes,
    respectivamente.

\subsection{pidsim.core.error.ControlSystemsError}

    \begin{verbatim}
raise ControlSystemsError(error_message)\end{verbatim}
    
    Função que implementa a exceção básica que é provocada pelo software.
    
    Herda todos os parametros e atributos da exceção padrão do Python.

\subsection{pidsim.core.helpers.get\_time\_near}

    \begin{verbatim}
t = get_time_near(t, y, point)\end{verbatim}
    
    Função que retorna o tempo '$t$' do ponto mais próximo ao ponto
    desejado '$point$', presente nos vetores '$t$' (tempos) e '$y$'
    (amplitudes).
    
\subsection{pidsim.core.pade.Pade\{1,5\}}

    \begin{verbatim}
g = Pade1(t)
g = Pade2(t)
g = Pade3(t)
g = Pade4(t)
g = Pade5(t)\end{verbatim}
    
    Funções que implementam a aproximação de Padè, para $1^a$, $2^a$, $3^a$,
    $4^a$ e $5^a$ ordens.
    
    Recebem um tempo de atraso de transporte desejado e retornam uma
    função de transferencia, para ser multiplicada com a função de transferencia
    original.

\subsection{pidsim.core.pid.identification}
    
    Módulo Python que implementa os métodos de identificação de processos.
    
    \subsubsection{pidsim.core.pid.identification.IdentificationMethod}
        
        \begin{verbatim}
IdentificationMethod(t, y)\end{verbatim}
        
        Classe base, herdada pelas outras classes, que implementam as
        características especificas de cada método.
        
        \textit{t} é o vetor dos tempos e \textit{y} é o vetor das amplitudes.
        Ambos são retornados pelo método de discretização escolhido pelo
        usuário.
    
    \subsubsection{pidsim.core.pid.identification.*}
        
        \begin{verbatim}
Alfaro(t, y)
Broida(t, y)
ChenYang(t, y)
Ho(t, y)
Smith(t, y)
Viteckova(t, y)\end{verbatim}
        
        Sub-classes da classe \textbf{pidsim.core.pid.identification.IdentificationMethod}.
        Implementam cada método de identificação suportado pelo software.

\subsection{pidsim.core.pid.tuning}
    
    Módulo Python que implementa os métodos de sintonia de controladores
    \acs{PID}.
    
    \subsubsection{pidsim.core.pid.identification.TuningMethod}
        
        \begin{verbatim}
TuningMethod(t, y, i_method)\end{verbatim}
        
        Classe base, herdada pelas outras classes, que implementam as
        características especificas de cada método de sintonia.
        
        \textit{t} é o vetor dos tempos, \textit{y} é o vetor das amplitudes.
        Ambos são retornados pelo método de discretização escolhido pelo
        usuário. \textit{i\_method} é o metodo de identificação desejado
        para simular o controlador
    
    \subsubsection{pidsim.core.pid.identification.*}
        
        \begin{verbatim}
ZieglerNichols(t, y, i_method)
CohenCoon(t, y, i_method)
ChienHronesReswick0(t, y, i_method)
ChienHronesReswick20(t, y, i_method)\end{verbatim}
        
        Sub-classes da classe \textbf{pidsim.core.pid.identification.TuningMethod}.
        Implementam cada método de sintonia \acs{PID} suportado pelo software.
